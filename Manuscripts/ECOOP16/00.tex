\title{%
  Formal Language Recognition with The Java Type Checker 
  \newline
  \color{red}{%
    \rmfamily\scshape Thou Mortal, Be Warned. \newline
    Thou Shallt Not Remove \newline
    This Commandment \newline
    While There Are Signs of Haste \newline
    in This Document!!!!\newline
  }
}

\documentclass[a4paper,USenglish]{lipics}
\usepackage{\jobname}

%\author{Tome Levy⏎
% Department of Computer Science⏎
% Technion---Israel Institute of Technology⏎
% \texttt{\small \href{mailto:stlevy@campus.technion.ac.il}{stlevy@campus.technion.ac.il}}}

\author{Anonymized for the submission}

\begin{document}

\maketitle
\begin{abstract}
  The term ‟Fluent API” has become a buzzword. So has the related term
‟DSL”. The objective of this research is to use the power of \Java generics
to combine the two. Specifically, to create a software system, to be called
\Self, that would make the definition of a DSL for fluent API as easy as the
writing an EBNF for this DSL\@. And, just like an ouroboros, the definition of this
EBNF will be through a fluent API generated by \Self itself.

\end{abstract}

\section{Introduction}
\section{Fluent APIs Java}
Ever after their inception\urlref{http://martinfowler.com/bliki/FluentInterface.html} \emph{fluent APIs}
  increasingly gain popularity~\cite{Hibernate:06,Freeman:Pryce:06,Larsen:2012} and research
  interest~\cite{Deursen:2000,Kabanov:2008}.
In many ways, fluent APIs are a kind of
  \emph{internal} \emph{\textbf Domain \textbf Specific \textbf Languages}~\cite{VanDeursen:Klint:2000,Hudak:1997,Fowler:2010}:
They make it possible to enrich a host programming language without changing it.
Advantages are many: base language tools (compiler, debugger, IDE, etc.) remain
  applicable, programmers are saved the trouble of learning a new syntax, etc.
However, these advantages come at the cost of expressive power;
  in the words of Fowler:
  ‟\emph{Internal DSLs are limited by the syntax and structure of your base language.}”†
  {M. Fowler, \emph{Language Workbenches: The Killer-App for Domain Specific Languages?},
    2005
    \newline
  \url{http://www.martinfowler.com/articles/languageWorkbench.html#InternalDsl}}.
Indeed, in languages such as \CC, fluent APIs
  often make extensive use of operator overloading (examine, e.g., \textsf{Ara-Rat}~\cite{Gil:Lenz:07}),
  but this capability is not available in \Java.

Despite this limitation, fluent APIs in \Java can be rich, expressive,
and in many cases cab boost productivity in elegant, neat code snippets such as

\begin{quote}
  \javaInput[minipage,width=\linewidth,left=-2ex]{../Fragments/camel-apache.java.fragment}
\end{quote}

(a use case of Apache Camel~\cite{Ibsen:Anstey:10}, open-source integration
framework), and,

\begin{quote}
  \javaInput[minipage,width=\linewidth,left=-2ex]{../Fragments/jOOQ.java.fragment}
\end{quote}

(a use case of jOOQ\urlref{http://www.jooq.org}, a framework for writing SQL
like code in \Java, much like LINQ project~\cite{Meijer:Beckman:Bierman:06} in
the context of \CSharp).

Other examples of fluent APIs in \Java are abundant:
jMock~\cite{Freeman:Pryce:06},
Hamcrest\urlref{http://hamcrest.org/JavaHamcrest/},
EasyMock\urlref{http://easymock.org/},
jOOR\urlref{https://github.com/jOOQ/jOOR},
jRTF\urlref{https://github.com/ullenboom/jrtf}
and many more.

The actual implementation of these many examples is traditionally not carried
out in the neat manner it could possibly take. Our reason for saying this is
that the fundamental problem in fluent API design is the decision on the
‟language”. This language is the precise definition of which sequences of
method applications are legal and which are not.

As it turns out, the question of whether a BNF definition of such a language
can be ‟compiled” automatically into a fluent API implementation is, a
question of the computational power of the underlying language. In \Java, the problem
is particularly interesting since in \Java (unlike e.g., \CC~\cite{Gutterman:2003}),
the type system is (probably) not Turing complete.

The question is then, what can be computed, and what can not be computed by
coercing the type system and the type checker of a certain programming language
to do abstract computations it was never meant to carry out? And, why should we
care?

The previous jOOQ example suggests that jOOQ imitates SQL, but, is it possible at
all to produce a fluent API for the entire SQL language, or XPath, HTML,
regular expressions, BNFs, EBNFs, etc.?

Of course, with no operator overloading it is impossible
to fully emulate tokens; method names though make a good substitute for tokens, as done
in
\begin{quote}
  \javaInput[minipage,width=45ex,left=-2ex]{../Fragments/jOOQ-mini.java.fragment}.
\end{quote}

The questions that motivate this research are:
\begin{quote}
  \begin{itemize}
    \item Given a BNF specification of a DSL, determine whether there exists a
      fluent API in \Java that can be made for this specification?

    \item In the cases that such fluent API is possible, can it be produced
      automatically?

    \item Is it feasible to produce a \emph{compiler-compiler} such as
      Bison~\cite{Bison:manual} to convert such language specification into a
      fluent API?

\end{itemize}
\end{quote}

Inspired by the theory of formal languages and automata,
  this study explores what can be done with fluent APIs in \Java.

\section{A Brief Review of Fajita - our Main Goal}
Consider \Cref{figure:sql-bnf}, a BNF specification of a fluent API for a certain
fragment of SQL.

\begin{figure}[H]
  \caption{\label{figure:sql-bnf}
    A BNF for a fragment of SQL select queries.
  }
  \begin{Grammar}
    \begin{aligned}
      \<Query> & \Derives \cc{select()} \<Quant>\~\cc{from(Table.\kk{class}\!\!\!)} \<Where> \hfill⏎
      \<Quant> & \Derives \cc{all()} \hfill⏎
                  & \| \cc{columns(Column[].\kk{class}\!\!\!)} \hfill⏎
      \<Where> & \Derives \cc{where()} \cc{column(Column.\kk{class}\!\!\!)} \<Operator> \hfill⏎
                  & \|ε \hfill⏎
      \<Operator> & \Derives \cc{equals(Expr.\kk{class}\!\!\!)}\hfill⏎
                  & \| \cc{greaterThan(Expr.\kk{class}\!\!\!)} \hfill⏎
                  & \|\cc{lowerThan(Expr.\kk{class}\!\!\!)} \hfill
    \end{aligned}
  \end{Grammar}
\end{figure}

To create a \Java implementation that realizes this fluent API,
  the designer feeds the grammar to \Fajita, as in
  \cref{figure:sql-bnf-java}.

\begin{figure}[H]
  \caption{\label{figure:sql-bnf-java}
    A \texorpdfstring{\Java}{Java} code excerpt defining the BNF specification of the fragment SQL
    language defined in \cref{figure:sql-bnf}.}
  \javaInput[minipage,width=\linewidth,left=-6ex]{sql.bnf.listing}
\end{figure}

We see that \Fajita's API is fluent in itself, and the
  call chain in \cref{figure:sql-bnf-java}, is structured almost
  exactly as in derivation rules in \cref{figure:sql-bnf}.
In particular, the code in \cref{figure:sql-bnf-java} shows how fluent API specification in \Fajita
  may include parameterless methods (\cc{select()}, \cc{all()} and \cc{where}) as well as methods which
  take parameters, e.g., method \cc{column} taking parameter of type \cc{Column} and
    method \cc{from} taking a \cc{Table} parameter.

Other than the derivation rules, \Fajita needs to be told the start rule
  and the sets of terminals and nonterminals.
These are specified in the first method call in the chain where,
  the enumerate types \cc{SQLTerminals} and \cc{SQLNonTerminals} are:

\begin{quote}
  \javaInput[minipage,width=\linewidth,left=-4ex]{sql.enums.listing}
\end{quote}

The call \cc{.go()}, occurring last in the chain, makes \Fajita generate types
and methods realizing the fluent API, in such a way that legal use of the API
like in \cref{figure:sql:legal} is syntactically legal,

\begin{figure}[H]
  \caption{\label{figure:sql:legal}
  Legal sequences of calls in the sql fragment example}
  \javaInput[minipage,width=\linewidth,left=-4ex]{sql.legal.listing}

\end{figure}

while snippets disobeying the BNF specification in \cref{figure:sql-bnf} like
\cref{figure:sql:illegal} , do not type check.

\begin{figure}[H]
  \caption{\label{figure:sql:illegal}
  Illegal sequences of calls in the sql fragment example}
  \javaInput[minipage,width=\linewidth,left=-4ex]{sql.illegal.listing}
\end{figure}

\section{The many trials of doing it in Java}
Fluent APIs are neat, but design is complicated,
involving theory of automata, type theory, language design, etc.

And still, automatic generation of these exist to some extent.
One example, is fluflu\urlref{https://github.com/verhas/fluflu}: a software artifact that uses
\Java annotations to define deterministic finite automata(henceforth, DFA), and then
compiles it to a fluent API based on this automaton. Usage example of fluflu is
depicted at \cref{figure:fluflu}.

\begin{figure}[ht]
  \caption{\label{figure:fluflu}
    Usage example of fluflu, generation a fluent
  API for the regular expression~\texorpdfstring{$a^*$}{a*}}
  \javaInput[minipage,left=-4ex]{fluflu.example.listing}
\end{figure}
The code excerpt in~\cref{figure:fluflu} defines a single stated DFA using
fluflu. The state is both initial and accepting, with a single self
transition, labeled with terminal~$a$. The regular language realized by the
automaton is~$L=a^*$.

Although fluent API generation was ‟achieved”, this example has many flaws, some of them are:

\begin{enumerate}
  \item The usage is complicated and the resulted code is messy.
  \item Defining a DFA is harder then writing, for example, the equivalent regular
        expression (and in some cases, the size of the DFA might be exponentially bigger)
  \item Languages defined by DFA can only be regular, a rather small
        class of languages.
\end{enumerate}

A question rises after seeing \Fajita and fluflu: how come the representation
of simple languages such as regular languages is so complex (as seen in fluflu)
and the representation of a more complex class of languages, is rather simple?
As it turns out, two factors are involved in the answer.

The first is the complexity of \emph{representing the language}.
Regular language are mostly defined by regular expression, but the language of
all regular expressions is actually context-free and not regular!
(Intuitively, since regular expressions uses parenthesizing). On the contrary,
the definition of context-free languages is usually done with a BNF (or a
context-free grammar), and the language of all BNFs is regular!

Thus, in order to define a fluent API for generating regular languages, one needs a
fluent API defined by a context-free language, while in order to define a fluent API for
context-free languages, only a regular fluent API is needed.

Since fluflu ‟knows” only how to generate fluent APIs for regular languages,
it cannot represent a fluent API for the generation of itself.
Respectively, since \Fajita's language is regular, and can generate context-free
  fluent APIs, \Fajita can generate itself.

The second factor is the \emph{practical complication} of generating the \Java types to
support the fluent API. As covered in \cref{section:example} the generation of
fluent APIs for regular languages is rather simple, while the generation of
those for context-free languages is the main challenge of this thesis.
\section{Contribution}
The main results of this research are:
\begin{quote}
  \begin{enumerate}
  \item If the DSL specification is that of a deterministic context-free
    language, then a fluent API exists for the language, but we do not know
    whether such a fluent API exists for more general languages.
  \par
  Recall that there are universal cubic time parsing
  algorithms~\cite{Cocke:1969,Earley:1970,Younger:1967} which can parse (and recognize) any
  context-free language. What we do not know is whether algorithms of this sort
  can be encoded within the framework of the \Java type system.
  \item
  There exists an algorithm to generate a fluent API that realizes any
  deterministic context-free languages. Moreover, this fluent API can create
  at runtime, a parse tree for the given language. This parse tree can then be
  supplied as input to the library that implements the language's semantics.
    \item
  Unfortunately, a general purpose compiler-compiler
  is not yet feasible with the current algorithm.
  \begin{itemize}
    \item One difficulty is usual in the fields of formal languages:
      The algorithm is complicated and relies on
      modules implementing complicated theoretical results, which, to the best of our
      knowledge, have never been implemented.
    \item Another difficulty is that a certain design decision in the
      implementation of the standard \texttt{javac} compiler is likely to make it choke on the
      \Java code generated by the algorithm.
  \end{itemize}
  \item
    We did implement a prototype for a compiler-compiler that works for LL(1) grammars.
  \end{enumerate}
\end{quote}

Other concrete contributions made by this work include
\begin{itemize}
  \item the understanding that the definition of fluent APIs is analogous to
      the definition of a formal language.
  \item a lower bound (deterministic pushdown automata)
    on the theoretical ‟computational complexity” of the \Java type system.
  \item an algorithm for producing a fluent API for deterministic context-free languages (even if impractical).
  \item a collection of generic programming techniques, developed towards this algorithm.
  \item a demonstration that the runtime of Oracle's \texttt{javac} compiler may be exponential in the program size.
  \item a new interpretation to the LL parsing algorithm, under a ‟realtime” constraint.
\end{itemize}

The theoretical result that any deterministic context-free grammar can be
automatically ‟compiled” to fluent API takes a toll of exponential blowup.
Specifically, the construction first builds a deterministic pushdown automaton
whose size parameters,~$g$ (number of stack symbols), and,~$q$ (number of
internal states), are polynomial in the size of the input grammar. This
automaton is then emulated by a weaker automaton, with as many as
\[
  O\left(g^{O(1)}\left(q²g^{O(1)}\right)^{qg^{O(1)}}\right)
\]
stack symbols.
This weaker automaton is then ‟compiled” into a collection of generic \Java types,
where there is at least one type for each of these symbols.


This work present an algorithm to compile an LL grammar of a fluent API
language into a \Java implementation whose size parameters are linear in
the size parameters of the LL parser generated by the classical
algorithm (\cref{algorithm:generation}) for computing LL parsers,
i.e., the performance loss due to implementation within the \Java
type checker is as small as we can hope it to be.

The savings are made possible by the use of a stronger automaton (the \RLLp,
described in detail below in \cref{section:realtime}) for the emulation, and
more efficient ‟compilation” of the \RLLp into the \Java type system.
We also present \Fajita
†{\itshape \textbf Fluent \textbf API for \textbf J\textsc{ava}
  (\textbf Inspired by the \textbf Theory of \textbf Automata)
}
a \Java tool that implements this algorithm.

\section{Related Work}

It has long been known
  that \CC templates are Turing complete in the following precise sense:

\begin{Proposition}
  \label{Theorem:Gutterman}
  For every Turing machine,~$m$, there exists a \CC program,~$Cₘ$ such that
    compilation of~$Cₘ$ terminates if and only if
      Turing-machine~$m$ halts.
      Furthermore, program~$Cₘ$ can be effectively generated from~$m$~\cite{Gutterman:2003}.
\end{Proposition}

Intuitively, this is due to the fact that templates in \CC
  feature both recursive invocation and conditionals (in the form of
  ‟\emph{template specialization}”).

In the same fashion, it should be mundane to make the judgment that
  \Java's generics are not Turing-complete since they offer no conditionals.
Still, even though there are time complexity results regarding type systems in functional
  languages, we failed to find similar claims for \Java.

Specialization, conditionals, \kk{typedef}s and other features of \CC templates,
  gave rise to many advancements in template/generic/generative programming
  in the language~\cite{Austern:98,Musser:Stepanov:1989,Backhouse:Jansson:1999,Dehnert:Stepanov:2000},
  including e.g., applications in numeric libraries~\cite{Veldhuizen:95,Vandevoorde:Josuttis:02},
  symbolic derivation~\cite{Gil:Gutterman:98}
  and a full blown template library~\cite{Abrahams:Gurtovoy:04}.

Garcia et al.~\cite{Garcia:Jarvi:Lumsdaine:Siek:Willcock:03} compared
  the expressive power of generics in half a dozen major programming languages.
  In several ways, the \Java approach~\cite{Bracha:Odersky:Stoutamire:Wadler:98}
  did not rank as well as others.

Not surprisingly, work on meta-programming using \Java generics,
  research concentrating on other means for enriching the language,
  most importantly annotations~\cite{Papi:08}.

The work on SugarJ~\cite{Erdweg:2011} is only one of many other attempts
  to achieve the embedded DSL effect of fluent APIs by language extensions.

Suggestions for semi-automatic generation can be found in the work of
Bodden~\cite{Bodden:14} and on numerous locations in the web.  None of these
materialized into an algorithm or analysis of complexity.  However, there is a
software artifact (the previously reviewed
fluflu\urlref{https://github.com/verhas/fluflu}) that automatically generates a
fluent API that obeys the transitions of a given finite automaton.

The challenges of \Java generic programming were highlighted by Garcia et
al.~\cite{Garcia:Jarvi:Lumsdaine:Siek:Willcock:03} research on the expressive
power of generics in half a dozen major programming languages,
Indeed, unlike \CC~\cite{Austern:98,Musser:Stepanov:1989,
Backhouse:Jansson:1999, Dehnert:Stepanov:2000,Gil:Gutterman:98,Abrahams:Gurtovoy:04}, the literature on meta-programming with \Java
generics is minimal.


\textbf{Outline.}
The main result is presented in~\cref{Section:proof}.
Towards this,~\cref{Section:preliminaries}, the preliminaries, 
  recalls the central notions we rely on: DSL, fluent API,
  context free languages, pushdown automata, etc. 
At the end of this section, the accumulated vocabulary is used to state this
  result more formally.
This terminology is used again in \Cref{Section:related} to offer 
  a perspective on related work.
\Cref{Section:toolkit} then builds a small toolkit of idioms and techniques
  for programming generics in \Java.
This toolkit is used in the subsequent~\cref{Section:proof} for
  proving our main theoretical result.
On~\Cref{Section:compiler} we will discuss our main computational tool, 
  e.g., the \Java compiler, and its expressiveness.

\section{Reminders and Preliminaries}
\label{Section:preliminaries}
\subsection{Method Chaining \emph{vs.} Fluent API: Reminder and Terminology}
The pattern ‟invoke function on variable \cc{sb}”, specifically with
  a function named \cc{append}, occurs six times in the code in \cref{Figure:chaining}(a), designed
  to format a clock reading, given as integers hours, minutes and
  seconds.

\begin{figure}[H]
  \caption{\label{Figure:chaining}%
    Recurring invocations of the pattern ‟invoke function on the same
      receiver”, before, and after method chaining.
  }%
    \begin{tabular}{@{}cc@{}}%
  \begin{lcode}[minipage,width=44ex,box align=center]{Java}
String time(int hours, int minutes, int seconds) {¢¢
  final StringBuilder sb = new StringBuilder();
  sb.append(hours);
  sb.append(':');
  sb.append(minutes);
  sb.append(':');
  sb.append(seconds);
  return sb.toString();
}\end{lcode}
\hfill
&
\hspace{1ex}
  \begin{lcode}[minipage,width=44ex,box align=center]{Java}
String time(int hours, int minutes, int seconds) {¢¢
    return new StringBuilder()
      ¢¢.append(hours).append(':')
      ¢¢.append(minutes).append(':')
      ¢¢.append(seconds)
      ¢¢.toString();
}\end{lcode}
⏎
\textbf{(a)} before & \textbf{(b)} after
\end{tabular}
\end{figure}

Some languages, e.g., \Smalltalk offer syntactic sugar, called \emph{cascading}, 
  for abbreviating this pattern.
\emph{Method chaining} is a ‟programmer made” syntactic sugar serving the same purpose:
  If a method~$f$ returns its receiver, i.e., \kk{this},
  then, instead of the series of two commands: \mbox{\cc{o.$f$(); o.$g$();}}, clients can write
  only one: \mbox{\cc{o.$f$().$g$();}}.
  \cref{Figure:chaining}(b) is the method chaining
  (also, shorter and arguably clearer) version of
  \cref{Figure:chaining}(a).
It is made possible thanks to the designer of class \cc{StringBuilder} ensuring that all overloaded variants of
  of \cc{append} return their receiver.

The distinction between \emph{fluent API} and method chaining is the identity of the receiver:
In method chaining, all methods are invoked on the same object, whereas in fluent API
  the receiver of each method in the chain may be arbitrary.
Fluent API are more interesting for this reason.
Consider, for example, the following \Java fragment (drawn from JMock~\cite{Freeman:Pryce:06})
\begin{JAVA}
allowing (any(Object.class))
  ¢¢.method("get.*")
  ¢¢.withNoArguments();
\end{JAVA}
Let the return type of function \cc{allowing} be denoted by~$τ₁$ and let the
  return type of function \cc{method} be denoted by~$τ₂$.
Then, the fact that~$τ₁≠τ₂$ means that the set of methods that can be placed after the dot
  in the partial call chain
\begin{JAVA}
allowing(any(Object.class)).
\end{JAVA}
is distinct from the set of methods that can be placed after the dot in the partial call chain
\begin{JAVA}
allowing(any(Object.class)).method("get.*").
\end{JAVA}
This distinction makes it possible to design expressive and rich fluent APIs, in which a
  sequence ‟chained” calls is not only readable, but also robust, in the sense that the
  sequence is type correct only when it the sequence makes sense semantically.

\subsection{Context Free Languages and Pushdown Automata: Reminder and Terminology}
Each of the notions discussed here is probably common knowledge
 (see e.g.,~\cite{Hopcroft:book:2001,Linz:2001} for a text book description, 
 or~\cite{Autebert:97} for a systematic review).
The purpose here is to set a unifying common vocabulary.

Let~$Σ$ be a finite alphabet of \emph{terminals} (often called input characters or tokens).
A \emph{language} over~$Σ$
  is a subset of~$Σ^*$.
Keep~$Σ$ implicit henceforth.

A \emph{\textbf Nondeterministic \textbf Pushdown \textbf Automaton} (NPDA) is a device for language recognition,
  made of a nondeterministic finite automaton
  and a stack of unbounded depth of stack elements.
An NPDA begins execution with a single copy of the initial stack element on the stack.
In each step, the NPDA
  examines the next input token,
  the state of the automaton,
  and the top of the stack.
It then pops the top element from the stack, and nondeterministically chooses which actions of
  its transition function to perform:
  Consuming the next input token,
  moving to a new state,
  or, pushing stack elements to the stack.
Nondeterminism effectively means
  that any combination of these actions can be done.

The language recognized by an NPDA is the set of strings that it accepts,
  either by reaching an accepting state or by encountering an empty stack.

A \emph{\textbf Context-\textbf Free \textbf Grammar}(CFG) is a formal description of a language.
A CFG~$G$ has three components:~$Ξ$ a set of \emph{variables} (also called nonterminals),
  a unique \emph{start variable}~$ξ∈Ξ$, and a finite set of (production) \emph{rules}.
A rule~$r∈G$ describes the derivation of a variable~$ξ∈Ξ$ into
  a string of \emph{symbols}, where symbols are either terminals or variables.
Accordingly, rule~$r∈G$ is written as~$r=ξ→β$, where~$β∈\left(Σ∪Ξ\right)^*$.
This description is often called BNF\@.
The \emph{language} of a CFG is the set of strings of terminals (and terminals only)
  that can be derived from the start symbol, following any sequence of applications of the rules.
CFG languages include regular languages, and are strictly contained in the set
  of ‟context-sensitive” languages.

The expressive power of NPDAs and BNFs is the same:
  For every language defined by a BNF, there exists an NPDA that recognizes it.
Conversely, there is a BNF definition for any language recognized by some NPDA.

NPDAs run in exponential deterministic time.
Since our encoding surely will be deterministic, we need a simpler
  model, that runs in polynomial deterministic time.

\begin{Definition}[Deterministic Pushdown Automaton]
  \label{Definition:DPDA}
  \slshape
  A \emph{deterministic pushdown automaton} (DPDA) is a quintuple~$⟨Q,Γ,q₀,A,δ⟩$
  where~$Q$ is a finite set of
    \emph{states},~$Γ$ is a finite
  \emph{set of stack elements},~$q₀∈Q$ is the initial state,
  and~$A⊆Q$ is the \emph{set of accepting states} while~$δ$ is
  the \emph{partial state transition function}~$δ:Q⨉Γ⨉(Σ∪❴ε❵)↛Q⨉Γ^*$.
  \par
  A DPDA begins its work in the initial state with a single designated stack element residing on the stack.
  At each step, the automaton examines:~$σ$, the next input token, the current state~$q∈Q$, and the element~$γ∈Γ$ at the top of the stack. Based on the values of these, it decides how to proceed:
  \par
  If~$γ$ is undefined, i.e., the stack is empty, the DPDA accepts the input and halts.
  The same happens if~$q∈A$.
  \par
  Suppose that~$δ(q,γ,ε)≠⊥$ (in this case, the definition of a DPDA requires that~$δ(q,γ,a')=⊥$ for all~$σ'∈Σ$),
  and let~$δ(q,γ,ε)=(q',ζ)$.
  Then the automaton pops~$γ$
  and pushes the string of stack elements~$ζ∈Γ^*$ into the stack.
  If~$δ(q,γ,a)=(q',ζ)$, then the same happens, but the
  automaton also irrevocably consumes the token~$σ$.
  \par
  If~$δ(q,γ,ε)=δ(q,γ,a)=⊥$ the automaton rejects the input and stops.
\end{Definition}

As mentioned above, NPDA languages are the same as CFG languages.
It is convenient to speak also of DCFG languages, the \emph{deterministic context-free grammar} languages,
which are those context-free languages that are recognizable by a DPDA.

DCFG languages are a strict container of regular languages,
  and are strictly contained in CFG languages.
On the other hand, DPDA are easier to parse:
  Recognition of a DPDA language
  can be done in linear time and one pass.
  In contrast, CYK, one of the best algorithms for recognizing NPDA languages run in super-quadratic time~\cite{Younger:1967,Cocke:1969,Earley:1970}.

\begin{Definition}
  \label{Definition:realtime}
  \slshape
We say a DPDA supplies the \emph{realtime condition} if
  for all~$q∈Q$,~$γ∈Γ$ :~$δ(q,γ,ε)=⊥$ (i.e., undefined).
\end{Definition}

The last construct we remind, is the less known \emph{Simple Jump Deterministic Pushdown Automaton},
the definition relies on Olsder's jump-DPDA supplying condition (S)~\cite{Olsder:05}.

\begin{Definition}[Simple-Jump Deterministic Pushdown Automaton]
  \label{Definition:JDPDA}
  \slshape
  A \emph{simple-jump deterministic pushdown automaton} (jump-DPDA) is a quintuple~$⟨Q,Γ,q₀,A,δ⟩$
  where~$Q,Γ,q₀,A$ are exactly as in~\cref{Definition:DPDA},
  and~$δ$ is the
  \emph{partial state transition function}~$δ:Q⨉Γ⨉(Σ∪❴ε❵)↛(Q⨉Γ^*)∪(Q⨉❴E❵⨉Γ)$
  (E is a special symbol which means ‟erase”).
  \par
  A jump-DPDA begins its work like a general DPDA\@.
  \par
  Let~$σ$ be in~$Σ∪❴ε❵$,~$q∈Q$ the current state, and~$γ∈Γ$ be the top of the stack.
  If~$δ(q,γ,σ)=(q,ζ)$ then it acts as a DPDA\@.
  If~$δ(q,γ,σ)=(q,E,γ')$ then as in a DPDA,~$γ$ is popped, and the automaton
  moves to state~$q$, but in addition, stack elements are popped until
  popping a single~$γ'$.
  If there is no~$γ'$ in the stack, the result is undefined.
  \par
  If~$γ$ is undefined, i.e., the stack is empty, the automaton accepts the input and halts.
  The same happens if~$q∈A$.
  \par
  If~$δ(q,γ,ε)=δ(q,γ,a)=⊥$ the automaton rejects the input and stops.
\end{Definition}

A realtime simple-jump deterministic pushdown automaton is a jump-DPDA
  as defined in~\cref{Definition:JDPDA}
  that supplies~\cref{Definition:realtime}.

\subsection{Fluent APIs realizing Deterministic Context Free Grammars}
The following definition is pertinent.
Let~$\textsf{java}$ be a function that translates a terminal~$σ∈Σ$
into a call to a uniquely named function (with respect to~$σ$).
Let~$\textsf{java}(α)$, be the function
  that translates a string~$α∈Σ^*$ into a fluent API call chain.
If~$α=σ₁⋯σₙ∈Σ^*$, then⏎
  \mbox{\qquad\qquad}$\textsf{java}(α)=$~\cc{.}$\textsf{java}(σ₁)$\cc{().}~$⋯$~\cc{.}$\textsf{java}(σₙ)$\cc{()}⏎
For~$Σ=❴a,b,c❵$ we can define~$\textsf{java}(a)=\cc{a}$,~$\textsf{java}(b)=\cc{b}$, and,~$\textsf{java}(c)=\cc{c}$.
With these definitions,~$\textsf{java}(caba) = \cc{.c().a().b().a()}$.

\begin{theorem}\label{Theorem:Gil-Levy}
  Let~$D$ be a DPDA recognizing a language~$L⊆Σ^*$.
  Then, there exists a \Java type definition,~$J_D$ for types~\cc{L},~\cc{D} and other types with the following
  property: Consider the \Java command
  \[
    \cc{L_= M.build~$\textsf{java}(α)$}\cc{.\$()};
  \]
  (in words: starting from the \kk{static} field of \kk{class}~\cc{M},
  apply the sequence of call chain~$\textsf{java}(α)$, terminate with a call to the
  end symbol~\cc{\$()} and then assign to newly declared variable~‟\cc{\_}” of type~\cc{L}).
  Then, this command type checks against~$J_M$ if an only if~$α∈L$.
  Furthermore, program~$J_M$ can be effectively generated from~$M$.
\end{theorem}


\section{Statement of the Main Result}
\label{Section:result}
Let~$\textsf{java}$ be a function that translates a terminal~$σ∈Σ$
into a call to a uniquely named function (with respect to~$σ$).
Let~$\textsf{java}(α)$, be the function
  that translates a string~$α∈Σ^*$ into a fluent API call chain.
If~$α=σ₁⋯σₙ∈Σ^*$, then⏎
  \mbox{\qquad\qquad}$\textsf{java}(α)=$~\cc{.}$\textsf{java}(σ₁)$\cc{().}~$⋯$~\cc{.}$\textsf{java}(σₙ)$\cc{()}⏎
For~$Σ=❴a,b,c❵$ we can define~$\textsf{java}(a)=\cc{a}$,~$\textsf{java}(b)=\cc{b}$, and,~$\textsf{java}(c)=\cc{c}$.
With these definitions,~$\textsf{java}(caba) = \cc{.c().a().b().a()}$.

\begin{theorem}\label{Theorem:Gil-Levy}
  Let~$D$ be a DPDA recognizing a language~$L⊆Σ^*$.
  Then, there exists a \Java type definition,~$J_D$ for types~\cc{L},~\cc{D} and other types with the following
  property: Consider the \Java command
  \begin{equation}
    \label{Equation:result}
    \cc{L~$ℓ$ = M.build~$\textsf{java}(α)$}\cc{.\$();}
  \end{equation}
  (in words: starting from the \kk{static} field of \kk{class}~\cc{M},
  apply the sequence of call chain~$\textsf{java}(α)$, terminate with a call to the
  end symbol~\cc{\$()} and then assign to newly declared variable~‟\cc{\_}” of type~\cc{L}).
  Then, this command type checks against~$J_M$ if an only if~$α∈L$.
  Furthermore, program~$J_M$ can be effectively generated†{in the computability lingo:
    effective~$≈$‟~\emph{there is time
  bounded algorithm for carrying out the task}”.} from~$M$.
\end{theorem}

There are several, mostly minor, differences between the structure \Java code in \cref{Equation:result}
  and the examples of fluent API we saw above in \cref{Figure:DSL}:
\begin{description}
  \item[Prefix, i.e., the stating \cc{M.build}.]
    All variables and functions of \Java are defined within a class.
    Therefore, a call chain must start with an object (\cc{M.build} in \cref{Equation:result})
      or, in case of \cc{static} methods, with the name of a class.
    With an appropriate \cc{import} statement, this prefix can be eliminated.
  \item[Suffix, i.e., the terminal \cc{.\$()} call.]
    In order to know whether~$α∈L$ the automaton recognizing~$L$ must
      know when~$α$ is terminated.
  \item[Parametrized methods.]
\end{description}


\section{Related Work}
\label{Section:related}
Modern programming languages acquire high-level constructs
  at a staggering rate.
The imminent adoption of closures in \Java and \CC,
  the generators of \CSharp, and ‟concepts” in
  \CC are just a few examples.

A theoretical motivation for this work
  is the exploration of the computational
  expressiveness of such features.
For example, it is known (see e.g.,~\cite{Gutterman:2003}) that
  \kk{template}s in \CC are Turing complete in the following precise sense:

\begin{Theorem}
  \label{Theorem:Gutterman}
  For every Turing machine,~$m$, there exists a \CC program,~$Cₘ$ such that
    compilation of~$Cₘ$ of terminates if and only if
      Turing-machine~$m$ halts.
  Furthermore, program~$Cₘ$ can be effectively†{in the computability lingo:
    effective~$≈$‟~\emph{there is time
  bounded algorithm for carrying out the task}”.} generated from~$m$.
\end{Theorem}

Intuitively, the proof relies on the fact that \kk{template}s
  feature recursive invocation and conditionals (in the form of
  ‟\emph{template specialization}”).

In the same fashion, it is mundane to make the judgment that
  \Java's generics are not Turing-complete: all recursive calls
  in these are unconditional.
In a sense, this article shall give a lower bound on the
  expressive power of \Java generics in terms of the Chomsky hierarchy~\cite{Chomsky:1963}.
This objective is more precisely expressed in the following conjecture.

\cite{Mention all \CC template programming}, there are many classics, including the \cite{BOOST library} I think. 

Mention funny tricks with annotations to Java. There is \cite{Michael Earnst} from 
  Washington State university. He fought for more support for annotations 
  and built a system for implementing non

Mention work by \cite{Bracha} on non-standard type systems.  

Mention this German (?) guy, who did suggest something like ours.  

Anything else you can find.

\subsection{Type State}
The toilette seat problem may be amusing to some, but it is not contrived in
any way. In fact, there is huge body of research on the general topics of
\emph{type-states}. (See e.g., review articles such
as~\cite{Aldrich:Sunshine:2009,Bierhoff:Aldrich:2005}) Informally, an object
that belongs to a certain type (\kk{class} in the object oriented lingo), has
type-states, if not all methods defined in this object's class are applicable
to the object in all states it may be in.

A classical example of type-states is a file object: which can be in one of two
states: ‟open” or ‟closed”. Invoking a \cc{read()} method on the object is only
permitted when the file is in an ‟open” state. In addition, method \cc{open()}
(respectively \cc{close()}) can only be applied if the object is in the
‟closed” (respectively, ‟open”) state.

Objects with type states such as toilette seats and files are not rarities.
In fact, a recent study~\cite{Beckman:2011} estimates
  that about 7.2% of \Java classes define protocols, that can be interpreted as type-state.
Type-state pose two main challenges to software engineering
\begin{enumerate}
  \item \emph{\textbf{Identification.}}
    In the typical case, type-state
        receive little to no mention at all in the documentation.
    The identification problem is to find the implicit
    type state in existing JAVA: Given an implementation of a class
    (or more generally of a software framework),
    \emph{determine} which sequences of method calls are valid and which violate the
    type state hidden in the JAVA.
  \item \emph{\textbf{Maintenance and Enforcement.}}
    Having identified the type-states, the challenge is in automatically flagging out
      illegal sequence of calls that does not conform
      with the type-state, furthermore, with the
      evolution of an API, the challenge is in updating the type-state information,
      and the type checking of JAVA of clients.
\end{enumerate}


\section{Techniques of Type Encoding}
\label{Section:toolkit}
This section presents techniques of type encoding in \Java.
Some readers may prefer to skip through to the next section,
where these are employed in the proof of \Cref{Theorem:Gil-Levy}.

Let~$g:Γ→Γ$ be a function,
  where set~$Γ$ is finite.
We argue that~$g$ can
  be represented using the compile-time mechanism of \Java.
  \cref{Figure:unary-function} encodes such a partial function for~$Γ=❴γ₁,γ₂❵$, where~$g(γ₁)=γ₂$
  and~$g(γ₂)=⊥$, i.e.,~$g(γ₂)$ is undefined.

\begin{figure}[hbt]
  \caption{\label{Figure:unary-function}%
    Type encoding of the partial function~$g:Γ↛Γ$,
    defined by~$Γ=❴γ₁,γ₂❵$,~$g(γ₁)=γ₂$ and~$g(γ₂)=⊥$.
  }
  \begin{tabular}{@{}c@{}c@{}c@{}}
    \hspace{-7ex}
    \parbox[c]{0.26\linewidth}{%
      \input ../Figures/unary-function-classification.tikz
    }%
    &
    \hspace{-1ex}
    \parbox[c]{0.64\linewidth}{%
      \javaInput{gamma.listing}
    }%
    &
    \hspace{-18ex}
    \parbox[c]{0.84\linewidth}{%
      \javaInput{gamma-example.listing}
    }%
⏎
\textbf{(a)} type hierarchy & \textbf{(b)} implementation & \hspace{-62ex} \textbf{(c)} use cases
  \end{tabular}
\end{figure}

The type hierarchy depicted in~\cref{Figure:unary-function}(a) shows five classes:
Abstract class~\cc{Γ} represents the set~$Γ$, final classes~\cc{γ1},~\cc{γ2}
  that extend~\cc{$Γ$}, represent the actual members of the set~$Γ$.
The remaining two classes are private final class~\cc{¤} that stands for an error value,
  and abstract class~\cc{$Γ'$} that denotes the augmented set~$Γ∪❴\text{¤}❵$.
Accordingly, both classes~\cc{¤} and~\cc{$Γ$} extend~\cc{$Γ'$}.†{The use
  short names, e.g.,~\cc{$Γ$} instead of \cc{$Γ'.Γ$},
    is made possible by to an appropriate \kk{import} statement.
    For brevity, all \kk{import} statements are omitted.}

The full implementation of these classes is provided in~\cref{Figure:unary-function}(b)†{Remember that \Java admits Unicode characters in identifier names}.
This actual code excerpt should be placed as a nested class of some appropriate host class. Import statements are omitted, here and henceforth for brevity.

The use cases in~\cref{Figure:unary-function}(c) explain better
  what we mean in saying that function~$g$ is encoded in the type system:
  An instance of class~\cc{$γ$1} returns a value of type~\cc{$γ$2} upon
  method call~\cc{g()}, while
  an instance of class~\cc{$γ$2} returns a value of our~\kk{private}
  error type~\cc{$Γ'$.¤} upon the same call.

Three recurring idioms employed in~\cref{Figure:unary-function}(b) are:
\begin{enumerate}
  \item An~\kk{abstract} class encodes a set.
    Abstract classes that extend it encode
      subsets, while~\kk{final} classes encode set members.
  \item The interest of frugal management of name-spaces is served
    by the agreement that if a class~\cc{$X$}~\kk{extends} another class~\cc{$Y$}, then~\cc{$X$} is also defined
    as a~\kk{static} member class of~$Y$.
  \item Body of functions is limited to a single~\kk{return}~\kk{null}\cc{;} command.
      This is to stress that at runtime, the code does not carry out any useful or interesting computation,
      and the class structure is solely for providing compile-time type checks.
†{%
A consequence of these idioms is that the augmented class~\cc{$Γ'$} is visible to clients.
It can be made~\cc{private}. Just move class~\cc{$Γ$} to outside of~\cc{$Γ'$}, defying the second idiom.
}
\end{enumerate}

Having seen how how inheritance and overriding make possible
  the encoding of unary functions, we turn now to encoding higher arity functions.
With the absence of multi-methods, other techniques must be used.

Consider the partial binary function~$f: R⨉S↛Γ$, defined by
\begin{equation}
  \label{Equation:simple-binary}
  \begin{array}{ccc}
    R=❴r₁,r₂❵ & f(r₁,s₁)=γ₁ & f(r₂,s₁)=γ₁⏎
    S=❴s₁,s₂❵ & f(r₁,s₂)=γ₂ & f(r₂, s₂)=⊥
  \end{array}.
\end{equation}
A \Java type encoding of this definition of function~$f$
  is in~\cref{Figure:simple-binary}(a); use cases
    are in~\cref{Figure:simple-binary}(b).

\begin{figure}[hbt]
  \caption{\label{Figure:simple-binary}%
    Type encoding of partial binary function~$f: R⨉S↛Γ$,
    where~$R=❴r₁,r₂❵$,~$S=❴s₁,s₂❵$, and~$f$
  defined by~$f(r₁,s₁)=γ₁$,~$f(r₁,s₂)=γ₂$,~$f(r₂,s₁)=γ₁$, and~$f(r₂, s₂)=⊥$.}
    \begin{tabular}{cc}
      \hspace{-2.5ex}
      \parbox[c]{0.57\linewidth}{\javaInput{binary-function.listing}}
        &
      \hspace{-16ex}
      \parbox[c]{51ex}{\javaInput[minipage,left=-2ex,width=51ex]{binary-function-example.listing}}
⏎
\parbox{0.57\linewidth}{\textbf{(a)} implementation (except for classes~\cc{$Γ$},~\cc{$Γ'$},~\cc{$γ$1}, and~\cc{$γ$2} given above
in \cref{Figure:unary-function}).}
& \hspace{-5ex}\textbf{(b)} use cases⏎
    \end{tabular}
  \end{figure}

The figure shows that to compute~$f(r₁,s₁)$ at compile time we write~\cc{f.r1().s1()}
and that the fluent API call chain~\cc{f.r2().s2().g()} results in
  a compile time error since~$f(r₂, s₂)=⊥$.

Class~\cc{f} in the implementation sub-figure serves as
  the starting point of the little fluent API defined here.
The return type of~\kk{static} member functions~\cc{r1()} and~\cc{r2()}
  is the respective sub-class of class~\cc{R}:
The return type of function~\cc{r1()} is class~\cc{R.r1};
  the return type of function~\cc{r2()} is class~\cc{R.r2}.

Instead of representing set~$S$ as a class,
  its members are realized as methods~\cc{s1()} and~\cc{s2()} in class~\cc{R}.
These functions are defined as~\kk{abstract} with return type~\cc{$Γ$'}
  in~\cc{R}.
Both functions are overridden in classes~\cc{r1} and~\cc{r2},
   with the appropriate co-variant change of their return type,

It should be clear now that the encoding scheme presented
  in \Cref{Figure:simple-binary} can be generalized to functions
  with any number of arguments, provided that the domain and range sets are finite.
The encoding of sets of unbounded size require means for creating an unbounded
 number of types.
Genericity can be employed to serve this end, as we shall see next.

\Cref{Figure:stack-use-cases} shows some use cases of a type encoding of
  a stack of unbounded depth, yet can only store members of the set~$Γ$.
With type encoding these are precisely classes~\cc{$γ$1}
  and \cc{$γ$2} defined in \cref{Figure:unary-function}.

\begin{figure}[htb]
  \caption{\label{Figure:stack-use-cases}%
    Use cases of a compile-time stack data structure.
  }
  \javaInput[minipage]{stack-use-cases.listing}
\end{figure}

The figure demonstrates a stack that starts with five items in it.
These are popped in order. Just before popping the last item, its
  value is examined.
Trying then to pop from an empty stack, or to examine its top, ends with
  a compile time error.

The expression⏎
  \mbox{\qquad\qquad} \cc{Stack.empty.$γ$1().$γ$1().$γ$2().$γ$1().$γ$1()}⏎
represents the sequence of pushing the value~$γ₁$ into an
empty stack, followed by~$γ₁$,~$γ₂$,~$γ₁$, and, finally,~$γ₁$.
This expression's type is that of~\cc{\_1}:⏎
\mbox{\qquad\qquad} \cc{P<$γ$1,P<$γ$1,P<$γ$¢2,P<$γ$1,P<$γ$1,E>>>>>}.⏎
A recurring building block occurs in this type.
This is generic type~\cc{P}, \emph{short for ‟Push”}, which takes two parameters:
  \begin{enumerate}
    \item the \emph{top} of the stack, always a subtype of~\cc{$Γ$},
    \item the \emph{rest} of the stack, which can be of two kinds:
          \begin{enumerate}
            \item another instantiation of~\cc{P} (in most cases),
            \item non-generic type~\cc{E}, \emph{short for ‟Empty”}, that encodes the empty
                  set (only at the deepest~\cc{P}, rest is empty).
          \end{enumerate}
  \end{enumerate}
Incidentally, \kk{static} field \cc{Stack.bottom} is of type~\cc{E}.

\Cref{Figure:stack-encoding}(a) gives the type inheritance hierarchy of the \Java
implementation†{%
  unless otherwise stated, in saying implementation we mean actual
  code extract
}
implementation in~\cref{Figure:stack-encoding}(b).

\begin{figure}[htb]
  \caption{\label{Figure:stack-encoding} Type encoding of an unbounded
  stack containing members of~$Γ$ (as defined in \cref{Figure:unary-function}).}
    \begin{tabular}{cc}
      \parbox[c]{0.3\linewidth}{%
        \input ../Figures/stack-classification.tikz
      }
      &
      \hspace{-3ex}
      \parbox[c]{63ex}{\javaInput[minipage]{stack.listing}}⏎
      \textbf{(a)} type hierarchy & \textbf{(b)} implementation
    \end{tabular}
\end{figure}

We see that the ‟rest” parameter of generic type~\cc{P} must extend class \cc{Stack},
  and that both types~\cc{P} and~\cc{E} extend \cc{Stack}.
Other points to noice are:
\begin{itemize}
  \item The type at the top of the stack is precisely the return type of \cc{top()};
        it is overridden in~\cc{P} so that its return type is the first argument of~\cc{P}.
        The return type of \cc{top()} in~\cc{B} is the error value {$Γ'$.¤}.
  \item Pushing into the stack is encoded as functions~\cc{$γ$1()} and~\cc{$γ$2()};
        the two are overriden with appropriate covariant change of the return type in~\cc{P} and~\cc{E}.
        \par
        It is of some notice that the notion of covariance recurs:
          Overriding permits covariant change of the return type of a function,
          Similar covariant change is permitted in extending generic type extending another, e.g.,
          \begin{center}
            \javaInput[minipage]{mammal.listing}
        \end{center}
  \item Since empty stack cannot be popped, function \cc{pop()} is overriden in~\cc{E} to return
    the \emph{error} type \cc{Stack.¤}. This type is indeed a kind of a stack, except that each of the four stack
        functions: \cc{top()}, \cc{push()},~\cc{$γ$1()}, and,~\cc{$γ$2()}, return an appropriate error type.
\end{itemize}

It is natural now to ask whether this recursive generic type can give rise to S-expressions and a~$λ$-calculus.
Indeed, S-experssions can be encoded in the spirit of \cref{Figure:stack-encoding}, with a \cc{Cons} generic type 
with co-variant \cc{car()} and \cc{cdr)} methods. However, it is unlikely, 
  as shall be explained in the conclusions section. 


\section{The Jump-Stack Data-Structure}
\label{Section:jump}
We coninue on our journey to encode DPDA in \Java.
In this section we encode the tricky ‟jump” action defined in~\cref{Definition:JDPDA}.

A \emph{jump-stack} is a stack data structure whose elements are drawn from a finite set~$Γ$,
  except that jump-stack supports~$\textsf{jump}(γ)$,~$γ∈Γ$ operations,
    which means
  ‟repetetively pop elements from the stack up to and including the first occurrence of~$γ$”.
Let~$k=|Γ|$.

\begin{wrapfigure}[16]{r}{42ex}
  \caption{Skeleton of type encoding for the jump-stack data structure}%
  \label{Figure:jump}%
  \lstset{style=numbered}
  \javaInput[minipage,left=-2ex]{jump-stack.listing}
\end{wrapfigure}

\Cref{Figure:jump} shows the skeleton of type-encoding of a jump-stack whose
elements are drawn from type~\cc{$Γ$}
(\cref{Figure:unary-function}), i.e., either~\cc{$γ$1} or~\cc{$γ$2}.

Just like \cc{Stack} (\cref{Figure:stack-encoding}(b)),
  the generic type \cc{JS} which encodes jump-stacks, takes
  a \cc{Rest} parameter which is the type of a jump-stack after popping.
In addition \cc{JS} takes~$k$ type parameters, one for~$γ∈Γ$,
  which is the type encoding of the jump-stack after a~$\textsf{jump}(γ)$
  operation.
In the figure, there are two such parameters: \cc{J\_$γ$1}, and
  \cc{J\_$γ$2}.

Functions defined in \cc{JS} include not only the standard stack opertions: \cc{top},
\cc{pop()}, \cc{$γ1$()} and~\cc{$γ2$()} (encoding a push of~$γᵢ$,~$i=1,2$),
  but also functions \cc{jump\_$γ$1} and \cc{jump\_$γ$2},
  which encode~$\textsf{jump}(γᵢ)$
  thanks to the return type being~\cc{J\_$γ$i},~$i=1,2$.

The type hierarchy rooted at \cc{JS} is similar to that of
\cref{Figure:stack-encoding}(a):
  Two of the specializations are parameterless and are
  almost identical to their \cc{Stack}
  counterparts:
\cc{JS.E} encodes an empty jump-stack; \cc{JS.¤} encodes a jump-stack in error,
e.g., a after popping from \cc{JS.E}.

Type \cc{JS.P} (line 15 in the figure) makes the third specialization of \cc{JS}, representing
Type \cc{JS.P} (line 15 in \cref{Figure:stack-encoding}) makes the third specialization of \cc{JS}, encoding
  a stack with one or more elements.
Just like in \cref{Figure:id}, there are no overriden functions in \cc{JS.P}; it achieves
  its purpose by the parameters it takes and those it passes
  to the type it extends.

\begin{wrapfigure}[10]r{43ex}
  \caption{\label{Figure:jump-stack-push} Type \cc{JS.P} encoding a non-empty jump-stack}
  \javaInput[minipage,width=43ex,left=-2ex]{jump-stack-push.listing}
\end{wrapfigure}

Specifically, \cc{JS.P} takes
the same \cc{Top} and \cc{Rest} paramters (ll.16--17) as type \cc{Stack.P}:
  as well as~$k$ additional paramters:
  \cc{J\_$γ$1} and \cc{J\_$γ$2} (ll.18--18)
which are the types encoding the jump-stack
  after the executation~$\textsf{jump}(γᵢ)$,~$i=1,2$.
Type \cc{JP.P'} passes these four parameters
to type \cc{Pʹ} which it extends (l.21)
The fifth parameter to \cc{Pʹ} (l.22) is the current incarnation of~\cc{P}, i.e.,
  \cc{P<Top, Rest, J\_γ1, J\_γ2>}.

The auxliary type \cc{JS.Pʹ} itself is depicted in \cref{Figure:jump-stack}.
By extending type \cc{JS} and passing the correct \cc{Rest} (respectively \cc{J\_$γ$1}, \cc{J\_$γ$2})
  parameter to it, \cc{JS.Pʹ} inherits correct declaration of function \cc{pop()} (l.6)
  (respectively \cc{jump\_$γ$1} (l.10), \cc{jump\_$γ$2} (l.11)).

Extending type \cc{JS} and passing the correct \cc{Rest} parameter to it, 
\cc{JS.Pʹ} inherits a correct declaration of function \cc{pop()} (l.6~\cref{Figure:jump}) 

Use cases for the encoded jump stack data structure in presented in \cref{Figure:jump-stack-example}.

\begin{wrapfigure}[12]r{63ex}
  \caption{\label{Figure:jump-stack-example} Use cases for the~\cc{JS} type hierarchy}
  \javaInput[minipage,width=63ex,left=-2ex]{jump-stack-example.listing}
\end{wrapfigure}

\cc{\_1} represents a stack that has been pushed \cc{$\gamma$2}, \cc{$\gamma$1}, \cc{$\gamma$1}.
Its first type argument is the Top,~$\gamma1$.
The second and third arguments,~\cc{Rest} and~\cc{J_$\gamma$1} correspondingly, 
  are represented by the same type, since the top of the stack is type~\cc{$\gamma$1}
  former~$\textsf{jump}(γᵢ)$ pops only the topmost element.
The last argument is~\cc{JS.E} represents~\cc{J_$\gamma$2}, and truly,
  operation~$\textsf{jump}(γ_2)$ in that case causes popping the whole stack.
  
\cc{\_2} is the result of~$\textsf{jump}(γ_1)$ on~\cc{\_1}.
As we saw the fourth type argument of~\cc{\_1}, the result's type is~\cc{JS.E}

\section{Proof of \Cref{Theorem:Gil-Levy}}
\label{Section:proof}
On a first sight, the proof of \cref{Theorem:Gil-Levy} could follows the techniques
  sketched in \cref{Section:toolkit} to type encode a DPDA (\cref{Definition:DPDA}).
The partial transition function~$δ$ may be type encoded as in \cref{Figure:simple-binary},
and the stack data structure of a DPDA can be encoded as in \cref{Figure:stack-encoding}.

The techniques however fail with~$ε$-transitions,
  which allow the automaton to move between an unbounded number of
  configurations and maneuver the stack in a non-trivial manner,
  without making any progress on the input.
The fault in the scheme lies with compile time computation being carried out
  by the~$\textsf{java}(σ)()$ functions, each converting
  their receiver type to the type of the receiver of the next call in the chain.
We are not aware of a \Java type encoding which makes
  it possible to convert an input type into an output type, where
  the output is a computed from the input by an unbounded number o steps.†{%
    With the presumption that the \Java compiler halts for all inputs (a presumption that does
    not hold for e.g., \CC, and was never proved for \Java), the claim that there is no \Java type encoding of all DPDAs can be proved:
 Employing~$ε$-transitions, it is easy to construct an automaton~$A^∞$ that
  never halts on any input.
A type encoding of~$A^∞$ creates programs that send the compiler in an infinite loop.
}

The literature speaks of finite-delay DPDAs, in which the number
  of consecutive~$ε$-transitions is uniformly bounded and even of
  realtime DPDAs in which this bound is 0, i.e., no~$ε$-transitions.
Our proof relies on a special kind of realtime automata,
  described by Courcelle~\cite{Courcelle:77}.

\begin{Definition}[Simple-Jump Single-State Realtime Deterministic Pushdown Automaton]
  \label{Definition:JDPDA}
  \slshape
  A \textit{simple-jump, single-state, realtime deterministic pushdown automaton}
  (jDPDA, for short) is a triplet~$⟨Γ,γ₁,δ⟩$
  where~$Γ$ is a set of stack elements,~$γ₁∈Γ$ is the initial stack element,
  and~$δ$ is the \emph{partial transition function},~$δ:Γ⨉Σ↛Γ^*∪j(Γ)$,
  \[
    j(Γ) = ❴ \textit{instruction \textup{\textsf{jump}}}(γ) \; | \;γ∈Γ❵.
  \]
  A configuration of a jDPDA is some~$c∈Γ^*$ representing the stack contents.
  Initially,~$ζ=γ₁$.
  For technical reasons, assume that the input terminates with~$\$ \not∈Σ$, a special end-of-file character.
  \begin{itemize}
    \item At each steps a jDPDA examines~$σ∈Σ$,
          the next input character and~$γ$ the element at the top of the stack, and
          executes the following:
          \begin{quote}
            \begin{enumerate}
              \item consume~$σ$
              \item pop~$γ$
              \item if~$δ(γ,σ)=ζ$,~$ζ∈Γ^*$, the automaton pushes~$ζ$ into the stack.
              \item if~$δ(γ,σ)=\textsf{jump}(γ)$,~$γ∈Γ$, then the automaton repetitively
                    pops stack elements up-to and including the first occurrence of~$γ$.
            \end{enumerate}
          \end{quote}
    \item If the next character is~$\$$, then the automaton may reject or accept,
          depending on the value of~$γ$.
  \end{itemize}
  In addition, the automator rejects if~$δ(γ,σ) =⊥$, or if it encounters
  an empty stack (either at the beginning of a step or on course
  of a \textsf{jump operation}.
\end{Definition}

\begin{wraptable}r{29ex}
  \caption{\label{Table:A} The transition function of a jDPDA~$A$,~$Σ=❴σ₁,σ₂❵$,~$Γ=❴γ₁,γ₂❵$}
  \begin{tabular}{c|ll}
             & \hfill~$γ₁$ & \hfill~$γ₂$⏎
    \midrule
$σ₁$ & $\textsf{push}(γ₁,γ₁)$ & $⊥$⏎
$σ₂$ & $\textsf{push}(γ₂,γ₂)$ & $\textsf{jump}(γ₁)$⏎
$\$$ & $\textsf{reject}$ & $\textsf{accept}$⏎
  \end{tabular}
\end{wraptable}

As it turns out, every DCFG language is recognized by some jDPDA, and conversely, every language accepted by a jDPDA
  is DCFG~\cite{Courcelle:77}.
The proof of \cref{Theorem:Gil-Levy} is therefore reduced to type-encoding of a given jDPDA.
Towards this end, we employ the type-encoding techniques developed above, and, in particular, the jump-stack data structure (\cref{Figure:jump}).

The simple~$|Γ|=2$,~$|Σ|=2$) jDPDA~$A$ defined in \cref{Table:A} will serve as our running example.
Let~$L$ be the language~$L$ recognized by~$A$.%
†{
  Incidintally,
\[
  L = ❴ σ₁^{n₁}σ₂^{2m₁} σ₁^{n₂}σ₂^{2m₂} ⋯ σ₁^{n_r}σ₂^{2m_r} \; | \; r≥0∧∀r', 1≤r'≤r⇒\sum_{i=1}^{r'} nᵢ≥\sum_{i=1}^{r'} mᵢ
\]
which is clearly not-regular; the equivalence of DCFG and jDPDAs means also that there is an nonambigous, deterministic
BNF for~$L$; neither the BNF nor the above repesentation are material for the proof.
}

\subsection{Preparation}
Generation of a type encoding for a jDPDA starts with two empty types for sets~$L$,~$Σ^*$:
\begin{JAVA}
\end{JAVA}
Let~$k =|Σ|$.
A configuration is encoded by a generic type~\cc{C}.
Essentially,~\cc{C} is a representation of the stack,
  but~$2k+1$ other parameters are required:
\begin{itemize}
  \item \cc{Rest} a type encoding of the stack after a push
  \item \cc{J$γ$1},…,\cc{J$γ${}$k$},
    type encoding of the stack
    after~$\textsf{jump}{γ₁}$,…,$\textsf{jump}{γₖ}$,
  \item \cc{JR$γ$1}…\cc{JR$γ${}$k$},
    type encoding of \cc{Rest}
    after~$\textsf{jump}{γ₁}$,…~$\textsf{jump}{γₖ}$,
\end{itemize}
Actual arguments would be properly constrained, making sure that
  they are (the type version of) pointers into the actual stack.
In the running example,~\cc{C} is defined as
\begin{JAVA}
\end{JAVA}
This excerpt shows also classes~\cc{E} and~\cc{¤} which encode (as in \cref{Figure:jump})
  the empty and the error configurations.

Classes \cc{C$γ$1},…,\cc{J$γ${}$k$} all specializing~\cc{C}
  encode stacks whose top element is~$γ₁$,…,~$γₖ$.
In the running example,
\begin{JAVA}
\end{JAVA}


\section{\Java Compiler expressiveness}
\label{Section:compiler}
The \Java compiler is the main computational tool we use
  in this manuscript.
In particular, the \Java generics mechanism is what
  boosted our expressiveness from the trivial \emph{regular languages}
  set, to the practical, reasonable \emph{deterministic context free languages} set.

An interesting question that was raised during this research,
  is what the runtime complexity of the \Java compiler is.
The reason this question is interesting, is its implication
  on the computational expressiveness of type-encoding.

For example, if we recognized that the \Java parser spends
  linear time on its input, we could say that it's not
  likely that we can type-encode nondeterministic CFG\@.
The reason is the fact that the best known algorithms
  today for parsing general nondeterministic CFG,
  as mentioned in~\Cref{Section:preliminaries}, run in super quadratic time.
And it is highly unlikely that the type-checker of \Java incidentally
  found an algorithm that is practically perfect in big-O notation.

As we explored this venue, we discovered that the type-checker of
  \Java actually runs in exponential time.

Consider an encoding of an S-expression in type~\cc{Cons}
  defined in~\cref{Figure:compiler}.

\begin{wrapfigure}[6]r{27ex}
  \caption{\label{Figure:compiler} Encoding of an binary type tree}
  \javaInput[minipage,width=27ex,left=-2ex]{compiler.listing}
\end{wrapfigure}

Type \cc{Cons} takes two type parameters, \cc{Car} and \cc{Cdr} (denoting left and right branches).
Denote the return type of \cc{d()} is~$τ= \cc{Cons< Cons<Car, Cdr>, Cons<Car, Cdr> >}$.
Let~$σ$ denote the type of the \kk{this} implicit parameter to~\cc{d}.
Since~$τ= \cc{Cons<}σ,σ\cc{>}$, we have~$|τ|≥2|Σ|$,
  where the size of a type is measured, e.g., in number of characters in its textual representation.
In a chain of~$n$ calls to \cc{d()}
\begin{equation}
  \label{Equation:n}
  \cc{(Cons<?,?>(null)).}\overbrace{\cc{d().}⋯\cc{.d()}}^{\text{$n$ times}}\cc{;}
\end{equation}
the size of the resulting type is~$O(2ⁿ)$.

\begin{wrapfigure}r{43ex}%
  \begin{minipage}{43ex}
  \caption{\label{Figure:compile-empiric} Compilation time
    (sec†{measured on an Intel i5-2520M CPU @ 2.50GHz~$⨉$4, 3.7GB memory, Ubuntu 15.04 64-bit,javac 1.8.0\_66}%
    ) \emph{vs.}
      length of call chain.
}
  \gnuplotloadfile[terminal=pdf,terminaloptions={crop size 2.5in,1.5in color enhanced font ",8" linewidth 1}]{../Figures/kill.gnuplot}
\end{minipage}
\end{wrapfigure}%

\Cref{Figure:compile-empiric} shows on the doubly logarithmic plane that the runtime (on a Lenovo X220)
  of the ‟javac” compiler (version 1.8.0\_66) in face of a \Java program
  consisting of \cref{Equation:n} in its \cc{main()} and \cref{Figure:compiler}.
We see that this runtime is asymptotically exponential.
(In fact, a variation of the construction may lead to even super-exponential growth rate of the size of types.)

We however note that this exponential growth is due to a design decision of the compiler.
Had the compiler used a representation of types that allows sharing of,
The footprint of exponentially sized types created by can be made linear
  with appropriate sharing of constituent types.


\section{Conclusion}
\label{Section:zz}
The theoretical part of this work dwells on the proof of~\cref{Theorem:Gil:Levy}.
Its engineering part is concerned is
  a software implementation that would make JAVA
  such as in~\cref{Figure:fluent} generate
  the required \Java JAVA that realizes the
  defined grammar, so that JAVA such as
  found in~\cref{Figure:toilette:legal} is type-correct,
  whereas JAVA found in~\cref{Figure:toilette:illegal} is not.
†{%
An important side effect of the implementation is that IDEs with built-in JAVA
completion
 (found e.g., in Eclipse~†{\href{http://www.eclipse.org/}{Eclipse home page}} and IntelliJ~†{\href{https://www.jetbrains.com/idea/}{Intellij home page}})
 will assist the programmer in making a correct use of the API.
 }

Accordingly, the contribution of this work is double folded:
  gaining better understanding of the computational expressiveness of
  \Java generics and type hierarchy, and, a better tool
  for designing, experimenting with and perfecting fluent APIs.

How about EBNF\@? Reference perhaps to Anna Beckermans thesis.\cite{Tomer:also try to trace citations from wikipedia}
It can be done. Further research directions might be found in exploring this venue. 


\bibliographystyle{abbrv}\small
\bibliography{author-names,other-shorthands,journals-full,publishers-abbreviated,%
  conferences-abbreviated,%
  journals-abbreviated,%
  yogi-book,yogi-practice,GPCE,OOPSLA,00}
\end{document}

Processing programming languages
\begin{description}
  \item[Lexical analysis] - the first step of the process in which the character strings generated by the
  programmer are aggregated to the abstract tokens defined by the language designer.
  \item[Syntactical analysis (parsing) ] - the second step, in which the processed strings of tokens
  conform to the rules of a formal grammar defined by the language's BNF (or EBNF).
  \item[Semantical analysis] - the next step, usually performed in unison with the previous step,
  in which the legal token sequences are given their semantic meaning.
\end{description}
Specifically, the proposal is that API design of follows the footsteps of
Accordingly, the designer of a fluent API has to follow these three conceptual
steps.
First is the identification of the \emph{vocabulary}, i.e.,
the set of method calls including type arguments that may take part in the
fluent API\@.
In this fluent API example
\begin{JAVA}
allowing (any(Object.class))
  ¢¢.method("get.*")
  ¢¢.withNoArguments();
\end{JAVA}
then, there are three method calls, and the vocabulary has three items in it.
\begin{itemize}
  \item~$ℓ₁ = \cc{any(Class<?>)}$
  \item~$ℓ₂ = \cc{allowing($ℓ₁$)}$
  \item~$ℓ₃ = \cc{method(String)}$
  \item~$ℓ₄ = \cc{withNoArguments()}$
\end{itemize}
