\begin{theorem}\label{Theorem:Gil-Levy:2}
  Let~$A$ be a DPDA recognizing a language~$L⊆Σ^*$.
  Then, there exists a \Java type definition,~$J_A$ for types~\cc{L},~\cc{A} and
    other types such that the \Java command
  \begin{equation}
    \label{Equation:result}
    \cc{A.build~$\textsf{java}(α)$;}
  \end{equation}
  type checks against~$J_A$ if an only if there exists~$β∈Σ^*$ such
  that~$αβ∈L$.
  Furthermore, program~$J_A$ can be effectively generated from~$A$.
\end{theorem}

The value of~\cref{Theorem:Gil-Levy} is the following:
  a call chain type-checks if and only if it is a prefix
  of some legal sequence.
Alternatively, a call chain won't type-check if there is no
  continuation that leads to a legal word in~$L$.

The proof will resemble the proof of~\cref{Theorem:Gil-Levy}.
  We will provide a similar implementation for a jump-stack (see\cref{Definition:JDPDA}),
  that will not compile under illegal prefixes.
