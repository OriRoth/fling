\documentclass[nonatbib,preprint,numbers]{sigplanconf}

\authorinfo{Mr.\ U. N. Owen}{}{}

\title{%
\begin{flushright}
  \scriptsize\bfseries
  Software design is language design.⏎
    (And vice versa.)⏎
  \footnotesize\mdseries\itshape
   a programmers' proverb
\end{flushright}
  \Huge \Fajita⏎
  \huge \itshape \textbf Fluent \textbf API for \textsc{\textbf Java}⏎
  \LARGE (\textbf Inspired by the \textbf Theory of \textbf Automata)
}

\usepackage{\jobname}

\begin{document}

\maketitle

\begin{abstract}
  The term ‟Fluent API” has become a buzzword. So has the related term
‟DSL”. The objective of this research is to use the power of \Java generics
to combine the two. Specifically, to create a software system, to be called
\Self, that would make the definition of a DSL for fluent API as easy as the
writing an EBNF for this DSL\@. And, just like an ouroboros, the definition of this
EBNF will be through a fluent API generated by \Self itself.

\end{abstract}

Must include in a place that would not be missed.
\begin{itemize}
  \item \Prolog Unification. We investigated wildcards, intersection types and
    (the rather limited) type inference of Java, but failed to employ these to
    increase the expressive power of the computation model.
        \item~$aⁿbⁿcⁿ$
  \item Lina's reduction size : Assume a grammar~$G=⟨Σ,Ξ, P⟩$.
    \begin{enumerate}
      \item On the first step, we transform the grammar to a DPDA~$⟨Q,Γ⟩$
        (LL(1) - linear grow, LR(1) - exponential blowup)
      \item For an arbitrary~$k$ the size of stack symbols turns to~$Γ^k$.
      \item The number of states grow to~$Q⨉Γ^k$.
      \item The number of stack symbols grow to~$Γ^k⨉(Q²⨉Γ^{2k})^{Q⨉Γ^k}$
    \end{enumerate}

\end{itemize}

\section{Introduction}
\section{Fluent APIs Java}
Ever after their inception\urlref{http://martinfowler.com/bliki/FluentInterface.html} \emph{fluent APIs}
  increasingly gain popularity~\cite{Hibernate:06,Freeman:Pryce:06,Larsen:2012} and research
  interest~\cite{Deursen:2000,Kabanov:2008}.
In many ways, fluent APIs are a kind of
  \emph{internal} \emph{\textbf Domain \textbf Specific \textbf Languages}~\cite{VanDeursen:Klint:2000,Hudak:1997,Fowler:2010}:
They make it possible to enrich a host programming language without changing it.
Advantages are many: base language tools (compiler, debugger, IDE, etc.) remain
  applicable, programmers are saved the trouble of learning a new syntax, etc.
However, these advantages come at the cost of expressive power;
  in the words of Fowler:
  ‟\emph{Internal DSLs are limited by the syntax and structure of your base language.}”†
  {M. Fowler, \emph{Language Workbenches: The Killer-App for Domain Specific Languages?},
    2005
    \newline
  \url{http://www.martinfowler.com/articles/languageWorkbench.html#InternalDsl}}.
Indeed, in languages such as \CC, fluent APIs
  often make extensive use of operator overloading (examine, e.g., \textsf{Ara-Rat}~\cite{Gil:Lenz:07}),
  but this capability is not available in \Java.

Despite this limitation, fluent APIs in \Java can be rich, expressive,
and in many cases cab boost productivity in elegant, neat code snippets such as

\begin{quote}
  \javaInput[minipage,width=\linewidth,left=-2ex]{../Fragments/camel-apache.java.fragment}
\end{quote}

(a use case of Apache Camel~\cite{Ibsen:Anstey:10}, open-source integration
framework), and,

\begin{quote}
  \javaInput[minipage,width=\linewidth,left=-2ex]{../Fragments/jOOQ.java.fragment}
\end{quote}

(a use case of jOOQ\urlref{http://www.jooq.org}, a framework for writing SQL
like code in \Java, much like LINQ project~\cite{Meijer:Beckman:Bierman:06} in
the context of \CSharp).

Other examples of fluent APIs in \Java are abundant:
jMock~\cite{Freeman:Pryce:06},
Hamcrest\urlref{http://hamcrest.org/JavaHamcrest/},
EasyMock\urlref{http://easymock.org/},
jOOR\urlref{https://github.com/jOOQ/jOOR},
jRTF\urlref{https://github.com/ullenboom/jrtf}
and many more.

The actual implementation of these many examples is traditionally not carried
out in the neat manner it could possibly take. Our reason for saying this is
that the fundamental problem in fluent API design is the decision on the
‟language”. This language is the precise definition of which sequences of
method applications are legal and which are not.

As it turns out, the question of whether a BNF definition of such a language
can be ‟compiled” automatically into a fluent API implementation is, a
question of the computational power of the underlying language. In \Java, the problem
is particularly interesting since in \Java (unlike e.g., \CC~\cite{Gutterman:2003}),
the type system is (probably) not Turing complete.

The question is then, what can be computed, and what can not be computed by
coercing the type system and the type checker of a certain programming language
to do abstract computations it was never meant to carry out? And, why should we
care?

The previous jOOQ example suggests that jOOQ imitates SQL, but, is it possible at
all to produce a fluent API for the entire SQL language, or XPath, HTML,
regular expressions, BNFs, EBNFs, etc.?

Of course, with no operator overloading it is impossible
to fully emulate tokens; method names though make a good substitute for tokens, as done
in
\begin{quote}
  \javaInput[minipage,width=45ex,left=-2ex]{../Fragments/jOOQ-mini.java.fragment}.
\end{quote}

The questions that motivate this research are:
\begin{quote}
  \begin{itemize}
    \item Given a BNF specification of a DSL, determine whether there exists a
      fluent API in \Java that can be made for this specification?

    \item In the cases that such fluent API is possible, can it be produced
      automatically?

    \item Is it feasible to produce a \emph{compiler-compiler} such as
      Bison~\cite{Bison:manual} to convert such language specification into a
      fluent API?

\end{itemize}
\end{quote}

Inspired by the theory of formal languages and automata,
  this study explores what can be done with fluent APIs in \Java.

\section{A Brief Review of Fajita - our Main Goal}
Consider \Cref{figure:sql-bnf}, a BNF specification of a fluent API for a certain
fragment of SQL.

\begin{figure}[H]
  \caption{\label{figure:sql-bnf}
    A BNF for a fragment of SQL select queries.
  }
  \begin{Grammar}
    \begin{aligned}
      \<Query> & \Derives \cc{select()} \<Quant>\~\cc{from(Table.\kk{class}\!\!\!)} \<Where> \hfill⏎
      \<Quant> & \Derives \cc{all()} \hfill⏎
                  & \| \cc{columns(Column[].\kk{class}\!\!\!)} \hfill⏎
      \<Where> & \Derives \cc{where()} \cc{column(Column.\kk{class}\!\!\!)} \<Operator> \hfill⏎
                  & \|ε \hfill⏎
      \<Operator> & \Derives \cc{equals(Expr.\kk{class}\!\!\!)}\hfill⏎
                  & \| \cc{greaterThan(Expr.\kk{class}\!\!\!)} \hfill⏎
                  & \|\cc{lowerThan(Expr.\kk{class}\!\!\!)} \hfill
    \end{aligned}
  \end{Grammar}
\end{figure}

To create a \Java implementation that realizes this fluent API,
  the designer feeds the grammar to \Fajita, as in
  \cref{figure:sql-bnf-java}.

\begin{figure}[H]
  \caption{\label{figure:sql-bnf-java}
    A \texorpdfstring{\Java}{Java} code excerpt defining the BNF specification of the fragment SQL
    language defined in \cref{figure:sql-bnf}.}
  \javaInput[minipage,width=\linewidth,left=-6ex]{sql.bnf.listing}
\end{figure}

We see that \Fajita's API is fluent in itself, and the
  call chain in \cref{figure:sql-bnf-java}, is structured almost
  exactly as in derivation rules in \cref{figure:sql-bnf}.
In particular, the code in \cref{figure:sql-bnf-java} shows how fluent API specification in \Fajita
  may include parameterless methods (\cc{select()}, \cc{all()} and \cc{where}) as well as methods which
  take parameters, e.g., method \cc{column} taking parameter of type \cc{Column} and
    method \cc{from} taking a \cc{Table} parameter.

Other than the derivation rules, \Fajita needs to be told the start rule
  and the sets of terminals and nonterminals.
These are specified in the first method call in the chain where,
  the enumerate types \cc{SQLTerminals} and \cc{SQLNonTerminals} are:

\begin{quote}
  \javaInput[minipage,width=\linewidth,left=-4ex]{sql.enums.listing}
\end{quote}

The call \cc{.go()}, occurring last in the chain, makes \Fajita generate types
and methods realizing the fluent API, in such a way that legal use of the API
like in \cref{figure:sql:legal} is syntactically legal,

\begin{figure}[H]
  \caption{\label{figure:sql:legal}
  Legal sequences of calls in the sql fragment example}
  \javaInput[minipage,width=\linewidth,left=-4ex]{sql.legal.listing}

\end{figure}

while snippets disobeying the BNF specification in \cref{figure:sql-bnf} like
\cref{figure:sql:illegal} , do not type check.

\begin{figure}[H]
  \caption{\label{figure:sql:illegal}
  Illegal sequences of calls in the sql fragment example}
  \javaInput[minipage,width=\linewidth,left=-4ex]{sql.illegal.listing}
\end{figure}

\section{The many trials of doing it in Java}
Fluent APIs are neat, but design is complicated,
involving theory of automata, type theory, language design, etc.

And still, automatic generation of these exist to some extent.
One example, is fluflu\urlref{https://github.com/verhas/fluflu}: a software artifact that uses
\Java annotations to define deterministic finite automata(henceforth, DFA), and then
compiles it to a fluent API based on this automaton. Usage example of fluflu is
depicted at \cref{figure:fluflu}.

\begin{figure}[ht]
  \caption{\label{figure:fluflu}
    Usage example of fluflu, generation a fluent
  API for the regular expression~\texorpdfstring{$a^*$}{a*}}
  \javaInput[minipage,left=-4ex]{fluflu.example.listing}
\end{figure}
The code excerpt in~\cref{figure:fluflu} defines a single stated DFA using
fluflu. The state is both initial and accepting, with a single self
transition, labeled with terminal~$a$. The regular language realized by the
automaton is~$L=a^*$.

Although fluent API generation was ‟achieved”, this example has many flaws, some of them are:

\begin{enumerate}
  \item The usage is complicated and the resulted code is messy.
  \item Defining a DFA is harder then writing, for example, the equivalent regular
        expression (and in some cases, the size of the DFA might be exponentially bigger)
  \item Languages defined by DFA can only be regular, a rather small
        class of languages.
\end{enumerate}

A question rises after seeing \Fajita and fluflu: how come the representation
of simple languages such as regular languages is so complex (as seen in fluflu)
and the representation of a more complex class of languages, is rather simple?
As it turns out, two factors are involved in the answer.

The first is the complexity of \emph{representing the language}.
Regular language are mostly defined by regular expression, but the language of
all regular expressions is actually context-free and not regular!
(Intuitively, since regular expressions uses parenthesizing). On the contrary,
the definition of context-free languages is usually done with a BNF (or a
context-free grammar), and the language of all BNFs is regular!

Thus, in order to define a fluent API for generating regular languages, one needs a
fluent API defined by a context-free language, while in order to define a fluent API for
context-free languages, only a regular fluent API is needed.

Since fluflu ‟knows” only how to generate fluent APIs for regular languages,
it cannot represent a fluent API for the generation of itself.
Respectively, since \Fajita's language is regular, and can generate context-free
  fluent APIs, \Fajita can generate itself.

The second factor is the \emph{practical complication} of generating the \Java types to
support the fluent API. As covered in \cref{section:example} the generation of
fluent APIs for regular languages is rather simple, while the generation of
those for context-free languages is the main challenge of this thesis.
\section{Contribution}
The main results of this research are:
\begin{quote}
  \begin{enumerate}
  \item If the DSL specification is that of a deterministic context-free
    language, then a fluent API exists for the language, but we do not know
    whether such a fluent API exists for more general languages.
  \par
  Recall that there are universal cubic time parsing
  algorithms~\cite{Cocke:1969,Earley:1970,Younger:1967} which can parse (and recognize) any
  context-free language. What we do not know is whether algorithms of this sort
  can be encoded within the framework of the \Java type system.
  \item
  There exists an algorithm to generate a fluent API that realizes any
  deterministic context-free languages. Moreover, this fluent API can create
  at runtime, a parse tree for the given language. This parse tree can then be
  supplied as input to the library that implements the language's semantics.
    \item
  Unfortunately, a general purpose compiler-compiler
  is not yet feasible with the current algorithm.
  \begin{itemize}
    \item One difficulty is usual in the fields of formal languages:
      The algorithm is complicated and relies on
      modules implementing complicated theoretical results, which, to the best of our
      knowledge, have never been implemented.
    \item Another difficulty is that a certain design decision in the
      implementation of the standard \texttt{javac} compiler is likely to make it choke on the
      \Java code generated by the algorithm.
  \end{itemize}
  \item
    We did implement a prototype for a compiler-compiler that works for LL(1) grammars.
  \end{enumerate}
\end{quote}

Other concrete contributions made by this work include
\begin{itemize}
  \item the understanding that the definition of fluent APIs is analogous to
      the definition of a formal language.
  \item a lower bound (deterministic pushdown automata)
    on the theoretical ‟computational complexity” of the \Java type system.
  \item an algorithm for producing a fluent API for deterministic context-free languages (even if impractical).
  \item a collection of generic programming techniques, developed towards this algorithm.
  \item a demonstration that the runtime of Oracle's \texttt{javac} compiler may be exponential in the program size.
  \item a new interpretation to the LL parsing algorithm, under a ‟realtime” constraint.
\end{itemize}

The theoretical result that any deterministic context-free grammar can be
automatically ‟compiled” to fluent API takes a toll of exponential blowup.
Specifically, the construction first builds a deterministic pushdown automaton
whose size parameters,~$g$ (number of stack symbols), and,~$q$ (number of
internal states), are polynomial in the size of the input grammar. This
automaton is then emulated by a weaker automaton, with as many as
\[
  O\left(g^{O(1)}\left(q²g^{O(1)}\right)^{qg^{O(1)}}\right)
\]
stack symbols.
This weaker automaton is then ‟compiled” into a collection of generic \Java types,
where there is at least one type for each of these symbols.


This work present an algorithm to compile an LL grammar of a fluent API
language into a \Java implementation whose size parameters are linear in
the size parameters of the LL parser generated by the classical
algorithm (\cref{algorithm:generation}) for computing LL parsers,
i.e., the performance loss due to implementation within the \Java
type checker is as small as we can hope it to be.

The savings are made possible by the use of a stronger automaton (the \RLLp,
described in detail below in \cref{section:realtime}) for the emulation, and
more efficient ‟compilation” of the \RLLp into the \Java type system.
We also present \Fajita
†{\itshape \textbf Fluent \textbf API for \textbf J\textsc{ava}
  (\textbf Inspired by the \textbf Theory of \textbf Automata)
}
a \Java tool that implements this algorithm.

\section{Related Work}

It has long been known
  that \CC templates are Turing complete in the following precise sense:

\begin{Proposition}
  \label{Theorem:Gutterman}
  For every Turing machine,~$m$, there exists a \CC program,~$Cₘ$ such that
    compilation of~$Cₘ$ terminates if and only if
      Turing-machine~$m$ halts.
      Furthermore, program~$Cₘ$ can be effectively generated from~$m$~\cite{Gutterman:2003}.
\end{Proposition}

Intuitively, this is due to the fact that templates in \CC
  feature both recursive invocation and conditionals (in the form of
  ‟\emph{template specialization}”).

In the same fashion, it should be mundane to make the judgment that
  \Java's generics are not Turing-complete since they offer no conditionals.
Still, even though there are time complexity results regarding type systems in functional
  languages, we failed to find similar claims for \Java.

Specialization, conditionals, \kk{typedef}s and other features of \CC templates,
  gave rise to many advancements in template/generic/generative programming
  in the language~\cite{Austern:98,Musser:Stepanov:1989,Backhouse:Jansson:1999,Dehnert:Stepanov:2000},
  including e.g., applications in numeric libraries~\cite{Veldhuizen:95,Vandevoorde:Josuttis:02},
  symbolic derivation~\cite{Gil:Gutterman:98}
  and a full blown template library~\cite{Abrahams:Gurtovoy:04}.

Garcia et al.~\cite{Garcia:Jarvi:Lumsdaine:Siek:Willcock:03} compared
  the expressive power of generics in half a dozen major programming languages.
  In several ways, the \Java approach~\cite{Bracha:Odersky:Stoutamire:Wadler:98}
  did not rank as well as others.

Not surprisingly, work on meta-programming using \Java generics,
  research concentrating on other means for enriching the language,
  most importantly annotations~\cite{Papi:08}.

The work on SugarJ~\cite{Erdweg:2011} is only one of many other attempts
  to achieve the embedded DSL effect of fluent APIs by language extensions.

Suggestions for semi-automatic generation can be found in the work of
Bodden~\cite{Bodden:14} and on numerous locations in the web.  None of these
materialized into an algorithm or analysis of complexity.  However, there is a
software artifact (the previously reviewed
fluflu\urlref{https://github.com/verhas/fluflu}) that automatically generates a
fluent API that obeys the transitions of a given finite automaton.

The challenges of \Java generic programming were highlighted by Garcia et
al.~\cite{Garcia:Jarvi:Lumsdaine:Siek:Willcock:03} research on the expressive
power of generics in half a dozen major programming languages,
Indeed, unlike \CC~\cite{Austern:98,Musser:Stepanov:1989,
Backhouse:Jansson:1999, Dehnert:Stepanov:2000,Gil:Gutterman:98,Abrahams:Gurtovoy:04}, the literature on meta-programming with \Java
generics is minimal.


\paragraph{Outline.}

\Cref{section:proposal}
  explains how \Fajita may serve the designer of a fluent API.
Then, \Cref{section:theoretical-background} discusses the
  the core computational theory problem that \Fajita
  needs to solve: recognizing and parsing of formal languages
  within the framework of the limited abilities of \Java
  generics.
The language recognition algorithm that \Fajita
  implements is the subject of \cref{section:intuition}.
The main ideas behind the bootstrapping definition of \Self
  are revealed in \Cref{section:bootstrapping}.
\Cref{section:zz} concludes.

\section{Type States and a Fluent API Example}
\label{section:example}
%! TEX root = 00.tex
The large body of research on the general topic of \emph{type-states} (see
e.g., these review articles~\cite{Aldrich:Sunshine:2009,Bierhoff:Aldrich:2005})
Informally, an object that belongs to a certain type (\kk{class} in the object
oriented lingo), has type-states, if not all methods defined in this object's
class are applicable to the object in all states it may be in.

File object is the classical example: It can be in one of two states: ‟open” or
‟closed”. Invoking a \cc{read()} method on the object is only permitted when
the file is in an ‟open” state. In addition, method \cc{open()} (respectively
\cc{close()}) can only be applied if the object is in the ‟closed”
(respectively, ‟open”) state.

A recent study~\cite{Beckman:2011} estimates that about~$7.2%$ of \Java
classes define protocols definable in terms of type-states.
This non-negligible prevalence raise two challenges:
\begin{enumerate}
  \item \emph{\textbf{Identification.}} Frequently, type-state receive little
    or no mention at all in the documentation. The challenge is in identifying
    the implicit type state in existing code.
    \par
    Specifically, given an implementation of a class (or more generally of a
    software framework), \emph{determine} which sequences of method calls are
    valid and which violate the type state requirement presumed by the
    implementation. \item \emph{\textbf{Maintenance and Enforcement.}} Having
    identified the type-states, the challenge is in automatically flagging out
    illegal sequence of calls that does not conform with the type-state.
    \par
    Part of this challenge is maintenance of these automatic flagging
    mechanisms as the type-state specification of the API evolves.
\end{enumerate}

\subsection{A Type State Example}

An object of type \cc{Seat}†
{%
  example inspired by earlier work of Richard Harter on the
  topic~\cite{Harter:05}.
}
is created in the \cc{down} state, but it can then be \cc{raise}d to the
\cc{up} state, and then be \cc{lower}ed to the \cc{down} state.
Such an object be used by two kinds of users, \cc{male}s and \cc{female}s, for
two distinct purposes: \cc{urinate} and \cc{defecate}.  


A fluent API enforcement of type states should signal the sequences of method
calls made in \cref{figure:toilette:illegal} as type errors. 

\begin{figure}[H]
  \begin{JAVA}
new Seat().male().raise().urinate();
new Seat().female().urinate();
  \end{JAVA}
  \caption{Legal sequences of calls in the toilette seat example}
  \label{figure:toilette:legal}
\end{figure}

At the same time, this type-state enforcement a fluent API should recognize the
sequences of \cref{figure:toilette:legal} as being 
  type correct.

\begin{figure}[H]
  \begin{JAVA}
new Seat().female().raise();
new Seat().male().raise().defecate();
new Seat().male().male();
new Seat().male().raise().urinate().female().urinate();\end{JAVA}
  \caption{Illegal sequences of calls in the toilette seat example}
  \label{figure:toilette:illegal}
\end{figure}

The protocol of a \cc{Seat} 

Method calls in 
It should be clear that the type checking engine of the compiler can
be employed to distinguish between legal and illegal sequences.
It should also be clear that fabricating the \kk{class}es, \kk{interface}s
and the \kk{extends} and \kk{implements} relationships between these, is
far from being trivial.

\subsection{Type State}
The toilette seat problem may be amusing to some, but it is not contrived in
any way.
  To illustrate, consider the toilette seat example.
In this example,
  there are a total of six methods that might be invoked.
\begin{quote}
  \begin{tabular}{lll}
    \cc{male()} & \cc{raise()} & \cc{urinate()}⏎
    \cc{female()} & \cc{lower()} & \cc{defecate()}⏎
  \end{tabular}
\end{quote}
A fluent API design specifies the order in which such calls can be made.

The \emph{first} novelty in this research is that the fluent API definition is
  through a CFG, written as a BNF.
\cref{figure:BNF} is such a specification for the toilette seat problem.

\begin{figure}[H]
  \begin{Grammar}
    \begin{aligned}
      \<Visitors> & ::= \<Down-Visitors> \hfill⏎
      \<Down-Visitors> & ::= \<Down-Visitor> \~\<Down-Visitors> \hfill⏎
      {} & \| \<Raising-Visitor> \~\<Up-Visitors> \hfill⏎
      {} & \| ε \hfill⏎
      \<Up-Visitors> & ::= \<Up-Visitor> \~\<Up-Visitors> \hfill⏎
      {} & \| \<Lowering-Visitor> \~\<Down-Visitors> \hfill⏎
      {} & \| ε \hfill⏎
      \<Up-Visitor> & ::= \cc{male()} \~\cc{urinate()} \hfill⏎
      \<Down-Visitor> & ::= \cc{female()} \~\<Action> \hfill⏎
                          & \| \cc{male()} \cc{defecate()} \hfill⏎
      \<Raising-Visitor> & ::= \cc{male()} \~\cc{raise()} \~\cc{urinate()} \hfill⏎
      \<Lowering-Visitor> & ::= \cc{female()} \~\cc{lower()} \~\<Action> \hfill⏎
                          & \| \cc{male()} \~\cc{lower()} \cc{defecate()} \hfill⏎
      \<Activity> & ::= \cc{urinate()} \hfill⏎
                          & \| \cc{defecate()} \hfill⏎
    \end{aligned}
  \end{Grammar}
  \caption{A BNF grammar for the toilette seat problem}
  \label{figure:BNF}
\end{figure}

\Self takes this grammar specification as input, and in response
  generates the corresponding
  \Java type hierarchy.

\subsection{Verbs and Nouns}
A second novelty of \Self is that the specification of a BNF such as in
  \cref{figure:BNF} can be also made with a \Java fluent API\@.
To do so, it is first necessary to
  define the set of \emph{grammar terminals}
  \begin{code}{JAVA}
enum ToiletteTerminals implements Terminal {¢¢
  male, female,
  urinate, defecate,
  lower, raise;
}
\end{code}
As common in fluent APIs we shall refer to these
as \emph{verbs}†{Admittedly, the words ‟male” and ‟female” are nouns.
  An excuse might be that the words are used as nouns to mean ‟male-visit” and ‟female-visit”.}.
Verbs are translated by \Self into methods.

We are also required to define the set of \emph{grammar variables}
\begin{code}{Java}
enum ToiletteVariables implements Variable {¢¢
  Visitors, Down_Visitors, Up_Visitors,
  Up_Visitor, Down_Visitor,
  Lowering_Visitor, Raising_Visitor,
  Activity
};
\end{code}
  We shall use the term ‟\emph{noun}” as synonymous to ‟variable”.

\subsection{Words and Grammar}
The terms ‟symbol” and ‟word” refer to an entity which is either
  a verb or a noun.

Once the verbs and the nouns are set, the grammar can be defined,
  using a fluent API generated by \Self itself as shown
  in \cref{figure:fluent}.

\begin{figure}[H]
  \begin{JAVA}[style=numbered]
new BNF()
  ¢¢.with(ToiletteTerminals.class)
  ¢¢.with(ToiletteSymbols.class)
  ¢¢.start(Visitors)
  ¢¢.derive(Visitors).to(Down_Visitors)
  ¢¢.derive(Down_Visitors)
    ¢¢.to(Down_Visitor).and(Down_Visitors)
    ¢¢.or(Raising_Visitor).and(Up_Visitors)
    ¢¢.orNone()
  ¢¢.derive(Up_Visitors)
    ¢¢.to(Up_Visitor).and(Up_Visitors)
    ¢¢.or(Lowering_Visitor).and(Down_Visitors)
    ¢¢.orNone()
  ¢¢.derive(Up_Visitor).to(male).and(urinate)
  ¢¢.derive(Down_Visitor)
    ¢¢.to(female).and(Action)
    ¢¢.or(male).and(defecate)
  ¢¢.derive(Raising_Visitor).to(male).and(raise).and(urinate)
  ¢¢.derive(Lowering_Visitor)
    ¢¢.to(female).and(lower).and(Action)
    ¢¢.or(male).and(lower).and(defecate)
  ¢¢.derive(Activity)
    ¢¢.to(urinate)
    ¢¢.or(defecate)
  ¢¢.go();
  \end{JAVA}
  \caption{A BNF grammar for the toilette seat problem}
  \label{figure:fluent}
\end{figure}

The call to function \cc{go()} (last line in \cref{figure:fluent}) instructs
  \Self to generate the code for the fluent API specified by the
  subsequent part of the expression.
Roughly speaking, nouns are translated to classes while verbs are translated to methods which
  take no parameters.
Two exceptions apply:
\begin{enumerate}
  \item
    Library classes such as \cc{String} and \cc{Integer}, just as user-defined
    classes such as \cc{Invoice} may be used as nouns.
    \Self generate class definitions only for classes whose name is declared
    in an \kk{enum} which is passed to \cc{with} verb in the BNF declaration.
  \item Verbs may take noun parameters, as explained below.
\end{enumerate}


\section{Type States with Pushdown Automata}
\label{section:generalization}
%! TEX root = 00.tex
Section: generalization:
    how to generate toilette with DFA\@.
  Generalization to pushdown.
Examples were pushdown automata become essential are: probably stack, scoping
in html, in hint only: nested SQL queries.

  \item
        Library classes such as \cc{String} and \cc{Integer}, just as user-defined
        classes such as \cc{Invoice} may be used as nouns.
        \Fajita generate class definitions only for classes whose name is declared
        in an \kk{enum} which is passed to \cc{with} verb in the BNF declaration.
  \item Verbs may take noun parameters, as explained below.
\end{enumerate}

As should be obvious from \cref{figure:fluent}, \Fajita will be implemented
in a bootstrapping fashion.
The specification of a BNF, is made itself using a fluent API.

This section describes how this is achieved.

\subsection{Reflective BNF}
\cref{figure:BNF:BNF} is a \emph{reflective BNF}:
It uses the notation introduced in \cref{figure:BNF}
to specify this same notation.

\begin{figure}
  \begin{Grammar}
    \begin{aligned}
      \<BNF>                & \Derives \<Header>\~\<Body>\~\<Footer> \hfill⏎
      \<Header>             & \Derives \<Variables> \~\<Terminals> \hfill⏎
      {}                    & \| \<Terminals> \~\<Variables> \hfill⏎
      \<Variables>          & \Derives \cc{with(Class<? \kk{extends} Variable>)}\hfill⏎
      \<Terminals>          & \Derives \cc{with(Class<? \kk{extends} Terminal>)}\hfill⏎
      \<Body>               & \Derives \<Start> \~\<Rules> \hfill⏎
      \<Start>              & \Derives \cc{start(\<Variable>)} \hfill⏎
      \<Rules>              & \Derives \<Rule> \~\<Rules>\hfill⏎
      {}                    & \| \<Rule> \hfill⏎
      \<Rule>               & \Derives \cc{derives(\<Variable>)} \<Conjunctions>\hfill⏎
      \<Conjunctions>       & \Derives \<First-Conjunction>\~\<Conjunctions>\hfill⏎
      \<First-Conjunction>  & \Derives \cc{to(\<Symbol>)}\~\<Symbols>\hfill⏎
      {}                    & \| \cc{toNone()}\hfill⏎
      \<Conjunctions> & \Derives \<Conjunction>\~\<Conjunctions>\hfill⏎
      {}                    & \| ε\hfill⏎
      \<Conjunction>  & \Derives \cc{or(\<Symbol>)}\~\<Symbols>\hfill⏎
      {}                    & \| \cc{orNone()} \hfill⏎
      \<Symbols>    & \Derives ε \hfill⏎
      {}                    & \| \cc{and(\<Symbol>)}\~\<Symbols> \hfill⏎
      \<Symbol>             & \Derives \cc{Variable} \hfill⏎
      {}                    & \| \<Verb>\hfill⏎
      {}                    & \| \<Verb>\~\cc{,} \<Noun> \hfill⏎
      \<Noun>               & \Derives \<Variable> \hfill⏎
      {}                    & \| \<Existing-Class> \hfill⏎
      \<Variable>           & \Derives \cc{Variable} \hfill⏎
      \<Footer>             & \Derives \cc{go()}\hfill⏎
    \end{aligned}
  \end{Grammar}
  \caption{A BNF grammar for defining BNF grammars}
  \label{figure:BNF:BNF}
\end{figure}
% this is not a formal BNF because <Symbol> and <Variable> are not defined in the Symbols set.
\begin{comment}
%%%%%%%%%%%%%%%%%%%%%%%%%%%
Note that this specification can only be approximate;
the figure uses verbs as replacement to indentation,
and special symbols such as~$|$,~$::-$ and~$ε$.
%%%%%%%%%%%%%%%%%%%%%%%%%%%
\end{comment}

From \cref{figure:BNF:BNF} we learn
that a BNF has three components: header, body and footer.
\begin{enumerate}
  \item The sets of terminals and variables are defined in the header part.
  \item The body starts with a definition of the start symbol, followed by a list of derivation
        rules.
  \item The footer is simply the verb \cc{go()} which instructs \Fajita
        to generate the code that realizes the fluent API specified by the grammar.
\end{enumerate}

A derivation rule starts with a variable, and is then followed by disjunctive alternatives.

The choice of verbs that may occur in, and between, these alternatives not incidental;
fluency was in mind:
\begin{description}
  \item[\cc{to}] to introduce the first symbol in the first conjunction.
  \item[\cc{or}] to introduce the first symbol in each subsequent conjunction.
  \item[\cc{and}] to introduce all but the first symbol in each such conjunction.
  \item[\cc{toNone}] to declare that the first conjunction is empty.
  \item[\cc{orNone}] to declare any subsequent conjunction is empty.
\end{description}

\subsection{Reflective BNF of fluent API}

To translate \cref{figure:BNF:BNF} into a fluent
API chain, the verbs and nouns must be defined.

Verb definitions are made in the code excerpt in
\cref{figure:Verbs}.

\begin{figure}[htb]
  \javaInput[minipage,width=\linewidth,left=-2ex]{../Fragments/fajita-verbs.fragment}
  \caption{The verbs of \Fajita}
  \label{figure:Verbs}
\end{figure}
Each of the enumerands in the figure is destined to be a
  name of a method in a class to be generated by \Fajita.

Noun definitions are made in the code excerpt in \cref{figure:Nouns}.

\begin{figure}[htb]
  \javaInput[minipage,width=\linewidth,left=-2ex]{../Fragments/fajita-nouns.fragment}
  \caption{The nouns of \Fajita}
  \label{figure:Nouns}
\end{figure}
  \Fajita will eventually generate a code with
  a class named after each the enumerands in the figure.

The enumerations \cc{BNFTerminals} and \cc{BNFVariables}
  are now employed in \cref{figure:BNF:fluent}.

\begin{figure}[htb]
  \javaInput[minipage,width=\linewidth,left=-2ex]{../Fragments/fajita-bootstrap.fragment}
  \caption{A BNF grammar for \Fajita API}
  \label{figure:BNF:fluent}
\end{figure}

The code excerpt in the figure is a rather long
sequence of method calls.
This fluent API sequence is a reflective BNF
of the \Fajita API;
indeed, we may check that \cref{figure:BNF:fluent} reiterates \cref{figure:BNF:BNF}
(with notational changes as appropriate).

\subsection{Parametrized Verbs}
The granularity of grammars of programming languages typically goes down to the \emph{lexical token} level,
but no deeper.
Such tokens, the building blocks of grammars, come in two flavors:
\begin{itemize}
  \item \emph{Monomorphic tokens} are tokens such as punctuation marks and
        certain keywords such as ‟\kk{if}”, ‟\cc{static}” and ‟\cc{class}”.
        Such tokens carry no information other than their mere presence.
  \item \emph{Polymorphic tokens} are tokens which carry content beyond
        presence (or absence). The prime example of these are identifiers.
\end{itemize}

This distinction applies also to fluent APIs:
Methods, or verbs, are the tokens, and a fluent APIs sequence consists of
method calls that come in two kinds: those that do not take parameters (such as \cc{toNone()} in \cref{figure:BNF:fluent}),
and those that do (such as the call \cc{derives($·$)} in the figure).

\Fajita supports verbs with, and without, noun parameters.
The following examples, drawn from \cref{figure:fluent} and \cref{figure:BNF:fluent},
demonstrate,
\begin{quote}
  \parbox[c]{30ex}{\javaInput{../Fragments/fajita-verbs-nouns.fragment}}
\end{quote}
(The above is made possible with minor \Java language trickery,
  involving overloading and use of variadic signatures,
  with respect to function name \cc{to}.
Same trickery was applied to verbs \cc{or}, and \cc{and}.)


\section{The Computational Model of \protect\Java Generics}
\label{section:background}
%! TEX root = 00.tex

\begin{enumerate}
  \item Gutterman
  \item Reference to emulation of DFA\@. We can only do
        a constant number of moves in each iteration.
  \item~$aⁿbⁿcⁿ$
  \item \Prolog Unification.
  \item We investigated wildcards, intersection types and
        (the rather limited) type inference of 
        Java, but failed to employ these to increase the expressive power of the computation model.
  \item Ecoop results.
  \item Lina reduction
\end{enumerate}

Lina's reduction size :Assume a grammar~$G=⟨Σ,Ξ,P⟩$.
    \begin{enumerate}
      \item On the first step, we transform the grammar to a DPDA~$⟨Q,Γ⟩$ (LL(1) - linear grow, LR(1) - exponential blowup)
      \item For an arbitrary~$k$ the size of stack symbols turns to~$Γᵏ$.
      \item The number of states grow to~$Q⨉Γᵏ$.
      \item The number of stack symbols grow to~$Γᵏ⨉(Q²⨉Γ^{2k})^{Q⨉Γᵏ}$
    \end{enumerate}

\endinput

% Taken from proof.tex
On a first sight, the proof of \cref{theorem:Gil-Levy} could follow the techniques
  sketched in \cref{section:toolkit} to type encode a DPDA (\cref{definition:DPDA}).
The partial transition function~$δ$ may be type encoded as in \cref{figure:simple-binary},
and the stack data structure of a DPDA can be encoded as in \cref{figure:stack-encoding}.

The techniques however fail with~$ε$-transitions,
  which allow the automaton to move between an unbounded number of
  configurations and maneuver the stack in a non-trivial manner,
  without making any progress on the input.
The fault in the scheme lies with compile time computation being carried out
  by the~$\Function java(σ)()$ functions, each converting
  their receiver type to the type of the receiver of the next call in the chain.
We are not aware of a \Java type encoding which makes
  it possible to convert an input type into an output type, where
  the output is computed from the input by an unbounded number of steps.
  †{With the presumption that the \Java compiler halts for all inputs (a presumption that does
    not hold for e.g., \CC, and was never proved for \Java), the claim that there is no \Java 
    type encoding for all DPDAs can be proved:
    Employing~$ε$-transitions, it is easy to construct an automaton~$A^∞$ that
    never halts on any input.
    A type encoding of~$A^∞$ creates programs that send the compiler in an infinite loop.
  }

The literature speaks of finite-delay DPDAs, in which the number
  of consecutive~$ε$-transitions is uniformly bounded and even of
  realtime DPDAs in which this bound is 0, i.e., no~$ε$-transitions.
Our proof relies on a special kind of realtime automata,
  described by Courcelle~\cite{Courcelle:77}.

\begin{Definition}[Simple-Jump Single-State Realtime Deterministic Pushdown Automaton]
  \label{definition:JDPDA}
  \slshape
  A \textit{simple-jump, single-state, realtime deterministic pushdown automaton}
  (jDPDA, for short) is a triplet~$⟨Γ,γ₁,δ⟩$
  where~$Γ$ is a set of stack elements,~$γ₁∈Γ$ is the initial stack element,
  and~$δ$ is the \emph{partial transition function},~$δ:Γ⨉Σ↛Γ^*∪j(Γ)$,
  \[
    j(Γ) = ❴ \textit{instruction \Function jump(\cdot)}(γ) \; | \;γ∈Γ❵.
  \]
  A configuration of a jDPDA is some~$c∈Γ^*$ representing the stack contents.
  Initially, the stack holds~$γ₁$ only.
  For technical reasons, assume that the input terminates with~$\$ \not∈Σ$, a special end-of-file character.
  \begin{itemize}
    \item At each step a jDPDA examines~$γ$,
    the element at the top of the stack,
    and~$σ∈Σ$, the next input character,
    and executes the following:
          \begin{quote}
            \begin{enumerate}
              \item consume~$σ$
              \item if~$δ(γ,σ)=ζ$,~$ζ∈Γ^*$, the automaton pops~$γ$, and pushes~$ζ$ into the stack.
              \item if~$δ(γ,σ)=\textsf{jump}(γ')$,~$γ'∈Γ$, then the automaton repetitively
                    pops stack elements up-to and including the first occurrence of~$γ'$.
            \end{enumerate}
          \end{quote}
    \item If the next character is~$\$$, the automaton may reject or accept (but nothing else),
          depending on the value of~$γ$.
  \end{itemize}
  In addition, the automaton rejects if~$δ(γ,σ) =⊥$ (i.e., undefined), or if it encounters
  an empty stack (either at the beginning of a step or on course of a \textsf{jump operation}).
\end{Definition}

\begin{table}[H]
  \caption{\label{table:A} The transition function of a jDPDA~$A$,~$Σ=❴σ₁,σ₂,σ₃❵$,~$Γ=❴γ₁,γ₂❵$ where~$γ₁$ is the initial element}
  \begin{centering}
    \begin{tabular}{c|ll}
      \toprule
              & \hfill~$γ₁$                     & \hfill~$γ₂$⏎
      \midrule
      $σ₁$ & $\textsf{push}(γ₁,γ₁,γ₂)$ & $\textsf{push}(γ₂,γ₂)$⏎
      $σ₂$ & \hfill$⊥$                        & $\textsf{push}(ε)$⏎
      $σ₃$ & \hfill$⊥$                        & $\textsf{jump}(γ₁)$⏎
      $\$$    & \hfill$\textsf{accept}$            & $\textsf{reject}$⏎
      \bottomrule
    \end{tabular}
  \end{centering}
\end{table}

As it turns out, every DCFG language is recognized by some jDPDA, and
conversely, every language accepted by a jDPDA is a DCFG
language~\cite{Courcelle:77}.  The proof of \cref{theorem:Gil-Levy} is
therefore reduced to type-encoding of a given jDPDA\@.  Towards this end, we
employ the type-encoding techniques developed above, and, in particular, the
jump-stack data structure (\cref{figure:jump}).

\subsection{Main Types}
Generation of a type encoding for a jDPDA starts with two empty types for sets~$L$,~$Σ^*$,
  where~$L$ represents the languages accepted by the jDPDA and~$Σ^*$ represents all words:
\begin{quote}
  %\javaInput[minipage,width=53ex,left=-2ex]{proof.headers.listing}
\end{quote}
(The full type encoding is in \cref{figure:A} below; to streamline the reading, we bring
  excerpts as necessary.)

A configuration is encoded by a generic type~\cc{C}.
Essentially,~\cc{C} is a representation of the stack,
  but~$k+1$ type parameters are required:
\begin{itemize}
  \item \cc{Rest}, a type encoding of the stack after a pop (or \textsf{jump} with the top element), and,
  \item $k$ types, named \cc{JR$γ$1}, … ,\cc{JR$γ${}$k$}, encoding the type of \cc{Rest}
    after~$\textsf{jump}(γ₁)$,…~$\textsf{jump}(γₖ)$.
\end{itemize}

Note that these~$k+1$ parameters are sufficient for describing a configuration,
  i.e., if the top is~$γⱼ$, then for all~$j≠i~$
\[
  \textsf{jump}(γⱼ) = \cc{Rest.}\textsf{jump}(γⱼ)
\]
In the special case of~$\textsf{jump}(γᵢ)$ the returned type is still~$\cc{Rest}$,
  this is due to the fact that before a \textsf{jump} operation,
  we do not pop an element from the stack.

All instantiations of~\cc{C} must make sure that actual parameters are properly constrained,
  to ensure that they are (the type version of) pointers into the actual stack,
  not a trivial task, as will be seen shortly.

In the running example,~\cc{C} is defined as:
\begin{quote}
  %\javaInput[minipage,width=61ex,left=-2ex]{proof.configuration.listing}
\end{quote}
This excerpt shows also classes~\cc{E} and~\cc{¤} which encode (as in \cref{figure:jump})
  the empty and the error configurations.

Type~\cc{C} defines~$ℓ+1$ functions (4 in the example), one for each possible input character,
  and one for the end-of-file character defined as \$.
Since~\cc{C} encodes an abstract configuration, return types of functions in it
  are the appropriate defaults which intentionally fail to emulate the automaton's execution.
  The return type of \cc{\$()} is \cc{ΣΣ} (rejection);
  the transition functions \cc{$σ$1()}, … \cc{$σ${}$ℓ$()}, return the raw type~\cc{C}.

\subsection{Top-of-Stack Types}

Types \cc{C$γ$1}, … ,\cc{C$γ${}$k$}, specializing~\cc{C},
  encode stacks whose top element is~$γ₁$, … ,$γₖ$.
In~$A$ there are two of these:
\begin{quote}
  %\javaInput[minipage,width=42ex,left=-4ex]{proof.many.listing}
\end{quote}

In~$A$, types \cc{C$γ$1} and \cc{C$γ$2} take three parameters;
in general ‟Top of Stack" types take the aforementioned~$k+1$ parameters.

\begin{figure*}
  \caption{\label{figure:chain} Accepting and non-accepting call chains with the
  type encoding of jDPDA~$A$ (as defined in \cref{table:A}).
  All lines in \cc{accepts()} type-check, while all lines in \cc{rejects()} do not type-check.}
  %\javaInput[minipage,left=-2ex,width=45ex]{proof.cases.listing}
\end{figure*}

The method signatures of these types are generated using the mentioned parameters.
The generating of methods will be discussed next.

The code defines the \kk{static} variable \cc{build}, the starting point
of all fluent API call chains, to be of type \cc{C$γ$1<E,¤,¤>}, i.e.,
  the starting configuration of the automaton is a stack whose top is~$γ₁$,
  and its \cc{Rest} parameter is empty (\cc{E}).
Any of the two jumps possible on this rest results with,~\cc{¤},
  an undefined stack.
Examples of accepting and rejecting call chains starting at \cc{A.build}
  can be seen in \cref{figure:chain}.

\subsection{Transitions}
It remains to show the type encoding of~$δ$,
  the transition function.
Overall, there are a total of~$k·(ℓ+1)$
  entries in a transition table such as \cref{table:A}.
Conceptually, these are encoded by selecting the correct return
  type of functions \cc{$σ$1()}, … ,\cc{$σ${}$k$()} and \cc{\$()} in each
  of the~$k$ ‟Top of Stack” types.
Thanks to inheritance, we need to do so only in the cases that this
  return type is different from the default.

Overall, there are six kinds of entries in a transition table:
\begin{description}

  \item[\textsf{reject}]
  The default return type of \cc{\$()} in~\cc{C} is \cc{$ΣΣ$}, which
  is \emph{not} a subtype of~\cc{L}. Normally the result of a call chain that ends with \cc{\$()}
  cannot be assigned to a variable of type~\cc{L}. Moreover, since \cc{$ΣΣ$} is \kk{private},
  there is little that clients can do with this result.

  \item[\textsf{accept}]
  The only case in which fluent call chain ending with \cc{\$()} can return
    type~\cc{L} is when the type returned of the call just prior to~\cc{.\$()} covariantly
    changes the return type of~\cc{\$()} to~\cc{L}.†{This is not to be confused with dynamic binding;
    types of fluent API call chains are determined statically.}
  \par
  Recall that a jDPDA can only accept after its input is exhausted.
  In \cref{table:A} we see that \textsf{accept} occurs when the top of the stack is~$γ₁$.
  We therefore add to the body of type \cc{C$γ$1} the line
  \begin{JAVA}
@Override L ¢\gobble$¢$();
  \end{JAVA}

  \item[$⊥$]
  When a prefix of the input is sufficient to conclude it must be rejected however it continues,
    the transition function returns~$⊥$.
  In~$A$ this occurs when the top of the stack is~$γ₁$ and one of~$σ_2$ or~$σ_3$ is read.
  To type encode~$δ(γ₁,σ₂) =⊥$, one must \emph{not} override \cc{$σ$2()} in type~\cc{C$γ$1};
    the inherited return type (l.15 \cref{figure:A}) is the raw~\cc{C}.
  Subsequent calls in the chain will all receive and return a raw~\cc{C}
    (Recall that all \cc{$σ${}$i$()},~$i=1,…,ℓ$, are functions in~\cc{C} that return a raw~\cc{C}).
  Therefore, the final \cc{\$()} will reject.
  \par
  Two other situations in which a jDPDA rejects but not demonstrated in~$A$ are:
    a \textsf{jump} that encounters an empty stack, and reading a character from when the stack is empty.
  In our type encoding these are handled by the special
    types~\cc{E} and~\cc{¤} (ll.17--18 ibid), both extend~\cc{C} without
    overriding any of its methods. Again, remaining part of the call chain will stick to
    raw~\cc{C}s up until the final \cc{\$()} call rejects the input.

  \item[$\textsf{jump}(γᵢ)$]
  The design of the generic parameters makes the implementation of~$\textsf{jump}(γᵢ)$
    operations particularly simple.
  All that is required is to covariantly change the return type of the
    appropriate \cc{$σ${}$j$()} function to the appropriate \cc{JR$γ{}i$} or~\cc{Rest} parameter
    (recall that a jump occurs after popping the current element from the stack, so % or is it now TOMER XXXXXXXXXXXXXXXXXX
    we refer to \cc{JR} type parameters rather than~\cc{J}'s).
  \par
  In \cref{table:A} we find that~$δ(γ₂,σ₃) =\textsf{jump}(γ₁)$. 
  Accordingly, the type of \cc{$σ$2()} in \cc{C$γ$2} (l.34) is \cc{JR$γ$1}.

  \item[$\textsf{push}(ζ)$]
  Push operations are the most complex, since they involve a pop of the top stack element,
    and pushing any number, including zero, of new elements.
  The challenge is in constructing the correct~$k+1$-parameter instantiation of~\cc{C},
    from the current parameters of the type.
  Each of these~$k+1$ is also an instantiation of~\cc{C} which may require more such
    parameters.
  Even though the number of ingredients is small, the resulting type expressions
    tend to be excessively long and unreadable.
  \par
  The predicament is ameliorated a bit by the idea,
    demonstrated above with auxiliary type~\cc{Pʹ}
    (\cref{figure:jump-stack-push}),
    of delegating the task of creating a complex type to an auxiliary
    generic type.
  The task of this sidekick is simplified if some of its generic
  parameters are sub-expressions that recur in the desired
  result.
  \par
  Cases in point
    are~$δ(γ₁,σ₁)=\textsf{push}(γ₁,γ₁,γ₂)$, and~$δ(γ₂,σ₁)=\textsf{push}(γ₂,γ₂)$ of \cref{table:A}.
  The corresponding sidekick types,
    (\cc{$γ$1$σ$1\_Push\_$γ$1$γ$1$γ$2} and \cc{$γ$2$σ$1\_Push\_$γ$2$γ$2})
    can be found in lines 36--43 of \cref{figure:A}.
  The first of these define the correct return type
    of \cc{$σ$1()} in case \cc{$γ$1} is the top element,
    the second of \cc{$σ$2}, in case \cc{$γ$2} is the top element.
  Examine now the definition of types \cc{C$γ$1},\cc{C$γ$2} in the figure,
    and in particular lines 21--23 and 29--31 which define the list of types they extend.
  Notice that each extends one of the sidekicks, inheriting the covariant
    overrides of \cc{$σ$1()}.
  \par
  More generally, economy of expression may require that for each case
    of~$δ(γ,σ)=\textsf{push}(ζ)$ in the transition table,
    one creates a sidekick type which overrides the appropriate \cc{$σ$()}
    function.
  The appropriate \cc{C$γ$} type then inherits the definition
    from the sidekick.
\end{description}

\begin{figure*}[htbp]
  \caption{\label{figure:A}Type encoding of jDPDA~$A$ (as defined in \cref{table:A})}
  %\javaInput[minipage,listing style=numbered,width=1.08\columnwidth]{proof.full.listing}
\end{figure*}

\paragraph*{Conclusion} The proof of~\Cref{theorem:Gil-Levy} is an algorithm, taking as input some jDPDA,
  and returning as output a set of \Java type definitions.
The returned types, allow a call chain~$\textsf{java}(\alpha)$,
  such that the type of the returned object represents the 
  configuration of the input automaton after reading~$\alpha$.
If the automaton rejects after~$\alpha$, then the returned type is the illegal~$\Sigma\Sigma$, 
  and if the automaton accpets, the type shall be~$L$.


% Taken from prefix.tex

\begin{Theorem}\label{theorem:Gil-Levy:2}
  Let~$A$ be a DPDA recognizing a language~$L⊆Σ^*$.
  Then, there exists a \Java type definition,~$J_A$ for types~\cc{L},~\cc{A},~\cc{C} and
    other types such that the \Java command
  \begin{equation}
    \label{equation:prefix:result}
    \cc{C c = A.build.$\textsf{java}(α)$;}
  \end{equation}
    type checks against~$J_A$ if an only if there exists~$β∈Σ^*$ such
    that~$αβ∈L$ and type \cc{C} is the configuration of~$A$ after reading~$α$.
  Furthermore, for any such~$β$,~\cref{theorem:Gil-Levy} applies such that the
  \Java command
  \begin{equation}
    \cc{L~$ℓ$ = A.build.$\textsf{java}(αβ)$}\cc{.\$();}
  \end{equation}
    always type-checks.
  Finally, the program~$J_A$ can be effectively generated from~$A$.
\end{Theorem}

Informally, a call chain type-checks if and only if it is a prefix
  of some legal sequence.
Alternatively, a call chain won't type-check if there is no
  continuation that leads to a legal string in~$L$.

The proof resembles~\cref{theorem:Gil-Levy}'s proof.
We provide a similar implementation for a jump-stack
  †{recall that the two formal constructs are of the same expressiveness},
  that will not compile under illegal prefixes.

The main difference between the two theorems is:
  in~\cref{theorem:Gil-Levy} we allowed illegal call chains to compile,
  but not return the required~\cc{L} type, while in~\cref{theorem:Gil-Levy:2}
  the illegal chain won't compile at all.

Since the code suggested by the proof highly resembles the previously
  suggested code, mainly the differences will be discussed.

We will use the same running example, defined by~\cref{table:A}.

\subsection{Main Types}
The main types here are a subset of the previously defined main types.

\begin{quote}
  %\javaInput[minipage,width=45ex,left=-2ex]{prefix-proof.configuration.listing}
\end{quote}

First, type \cc{$ΣΣ$} is removed.
A call chain that doesn't represent a valid prefix won't compile,
  thus, there is no need for an error return type such as \cc{$\SigmaΣ$}.
Second, \kk{interface}~\cc{C} is removed.
Without it, the configuration types won't have the
  methods \cc{$σ$1()}, … ,\cc{$σ${}$k$()} and \cc{\$()} from the supertype.
These inherited methods, is what differentiates the previous proof from the current.
Classes \cc{¤} and \cc{E} are defined similarly, except now they don't extend any type.

\subsection{Top-of-Stack Types}
Types \cc{C$γ$1}, … ,\cc{C$γ${}$k$}, still represent stacks
  with \cc{$γ$1}, … ,\cc{$γ${}$k$} as their top element,
  this time, the methods are defined ad-hock, in each type
  (they are not added in this figure as they are added with the use of sidekicks).
In~$A$ there are two such types:

\begin{quote}
  %\javaInput[minipage,width=51ex,left=-2ex]{prefix-proof.many.listing}
\end{quote}


Note, that the type parameters of the former types hasn't changed,
  since the model we are trying to implement, hasn't changed.
  These~$k+1$ parameters still suffice for our cause.

\begin{figure*}
  \caption{\label{figure:prefix-chain} Accepting and non-accepting call chains with the
  type encoding of jDPDA~$A$ (as defined in \cref{table:A}).
  All lines in \cc{accepts} type-check, and all lines in \cc{rejects}
  cause type errors}
  %\javaInput[minipage,width=42ex,left=-2ex]{prefix-proof.cases.listing}
\end{figure*}

In~\cref{figure:prefix-chain}, call chains in the~\cc{accepts()} method
  correctly type-checks (i.e., in~$L$), while the chains in~\cc{rejects()}
  do not type-check (i.e., these prefixes have no continuation that can lead to a legal word in~$L$),
  where the last method invocation generates an
\begin{quote}
  ‟\textsf{method~…~is undefined for the type~…~}”
\end{quote}
  error message.

The main difference between~\cref{figure:prefix-chain} and~\cref{figure:chain} is that there is no need to
  use an auxiliary function \cc{isL()} as in~\cref{figure:prefix-chain} since now illegal
  prefixes do not type-check.

\subsection{Transitions}
Due to the changes we expressed, the transition table is encoded slightly different.

Encoding of the legal operations \textsf{accept},~$\textsf{jump}(γᵢ)$ and~$\textsf{push}(ζ)$
  remains as in~\cref{theorem:Gil-Levy}, since we want the same behavior for legal call chains.
The minor differences are in the illegal operations \textsf{reject} and~$⊥$:

\begin{description}
 \item[\textsf{reject}] Since we add the methods ad-hock to each type, the reject entry means
   that the corresponding type, \emph{won't} have a~\cc{\$()} method, i.e., type~\cc{C$γ$2}
   doesn't have a method~\cc{\$()}.
 \item[$⊥$] We encounter~$⊥$ on the transition function when some input character~\cc{$σ$}
   is not allowed for the top of the stack element~$γ$. In that case, the corresponding type \cc{C$γ$}
   \emph{must not} have a method for~\cc{$σ$}, this way, invoking the methods will result in type error.
   In \cref{table:A} a~$⊥$ may occur when the top of the stack is~$γ₁$ and the input character is~$σ₂$,
   thus, no method \cc{$σ$2} is introduced in type~\cc{C$γ$1}.
\end{description}

The use of sidekicks is still allowed and recommended to improve readability of code.

\begin{figure*}
  \caption{\label{figure:prefix-A}Type encoding of jDPDA~$A$ (as defined in \cref{table:A})
    that allow a partial call chain, if and only if, there exists a legal continuation, that
    leads to a word in~$L$ (the language of~$A$}
  %\javaInput[minipage,listing style=numbered,width=\textwidth]{prefix-proof.full.listing}
\end{figure*}

\paragraph*{Conclusion}
In this section, a proof, similar to the one in~\cref{section:proof} is provided.
An algorithm was introduced, to not only emulate the running of some jDPDA~$A$,
  but also to ‟halt it” in the earliest time possible, i.e., only if there is
  no legal call chain from this point to result in a legal word in the language of~$A$.

% Taken from result.tex

Let~$\textsf{java}$ be a function that translates a terminal~$σ∈Σ$
into a call to a uniquely named function (with respect to~$σ$).
Let~$\textsf{java}(α)$, be the function
  that translates a string~$α∈Σ^*$ into a fluent API call chain.
  If~$α=σ₁⋯σₙ∈Σ^*$, then \[
  \textsf{java}(α)=\textsf{java}(σ₁)\cc{().}⋯\cc{.}\textsf{java}(σₙ)\cc{()}
\]
For example, when~$Σ=❴a,b,c❵$ let~$\textsf{java}(a)=\cc{a}$,~$\textsf{java}(b)=\cc{b}$, and,~$\textsf{java}(c)=\cc{c}$.
With these, \[
    \textsf{java}(caba) = \cc{c().a().b().a()}
  \]

\begin{Theorem}\label{theorem:Gil-Levy}
  Let~$A$ be a DPDA recognizing a language~$L⊆Σ^*$.
  Then, there exists a \Java type definition,~$J_A$ for types~\cc{L},~\cc{A} and
    other types such that the \Java command
  \begin{equation}
    \label{equation:result}
    \cc{L~$ℓ$ = A.build.$\textsf{java}(α)$}\cc{.\$();}
  \end{equation}
  type checks against~$J_A$ if an only if~$α∈L$.
  Furthermore, program~$J_A$ can be effectively generated from~$A$.
\end{Theorem}

\Cref{equation:result} reads: starting from the \kk{static} field \cc{build} of \kk{class}~\cc{A},
  apply the sequence of call chain~$\textsf{java}(α)$, terminate with a call to the
  ending character~\cc{\$()} and then assign to newly declared \Java variable~\cc{$ℓ$} of type~\cc{L}.

The proof of the theorem is by a scheme for encoding in \Java types
  the pushdown automaton~$A=A(L)$ that recognizes language~$L$.
Concretely, the scheme assigns a type~$τ(c)$
  to each possible configuration~$c$ of~$A$.
Also, the type of \cc{A.build} is~$τ(c₀)$, where~$c₀$ is the initial configuration of~$A$,

Further, in each such type the scheme places
  a function~$σ()$ for every~$σ∈Σ$.
Suppose that~$A$ takes a transition from configuration~$cᵢ$ to configuration~$cⱼ$
  in response to an input character~$σₖ$.
Then, the return type of function \cc{$σₖ$()} in type~$τ(cᵢ)$ is type~$τ(cⱼ)$.

With this encoding the call chain in \cref{equation:result}
  mimics the computation of~$A$, starting at~$c₀$ and ending with
  rejection or acceptance.
The full proof is in \cref{section:proof}.

Since the depth of the stack is unbounded, the number of configurations of $A$ is unbounded,
  and the scheme must generate an infinite number of types.
Genericity makes this possible, since a generic type is
  actually device for creating an unbounded number of types.

There are several, mostly minor, differences between the structure of the \Java code
in \cref{equation:result}
and the examples of fluent API we saw above, e.g., in \cref{figure:DSL}:
\begin{description}
  \item[Prefix, i.e., the starting \cc{A.build} variable.]
  All variables and functions of \Java are defined within a class.
  Therefore, a call chain must start with an object (\cc{A.build} in \cref{equation:result})
  or, in case of \cc{static} methods, with the name of a class.
  In fluent API frameworks this prefix is typically eliminated
  with appropriate \cc{import} statements.
  \par
  If so desired, the same can be done by our type encoding scheme: define all
  methods in type~$τ(c₀)$ as \cc{static} and \cc{import static} these.
  \item[Suffix, i.e., the terminal \cc{.\$()} call.]
  In order to know whether~$α∈L$ the automaton recognizing~$L$ must
  know when~$α$ is terminated.
  \par
  With a bit of engineering, this suffix can also be eliminated.
  One way of doing so is by defining type~\cc{L} as an \kk{interface}, and by making all types~$τ(c)$,~$c$ is
  an accepting configuration, as subtype of~\cc{L}.
  \item[Parameterized methods.]
  Fluent API frameworks support call chains with phrases such as:
  \begin{itemize}
    \item ‟\lstinline{.when(header(foo).isEqualTo("bar")).}”,
    \item ‟\lstinline{.and(BOOK.PUBLISHED.gt(date("2008-01-01"))).}”, and,
    \item ‟\lstinline{.allowing(any(Object.class)).}”.
  \end{itemize}
  while our encoding scheme assumes methods with no parameters.  
  \par
    Methods with parameters contribute to the user
      experience and readability of fluent APIs but their ``computational expressive power"' is the same.
      In fact, extending
      \cref{theorem:Gil-Levy} to support these requires these conceptually simple steps 
      \begin{enumerate}
        \item Define the structure of parameters to methods with appropriate fluent API, which may or
          may not be, the same as the fluent API of the outer chain, or the fluent API of parameters to
          other methods. Apply the theorem to each of these fluent APIs.
        \item
          If there are several overloaded versions of a method, consider each such version as a distinct
          character in the alphabet~$Σ$ and in the type encoding of the automaton.
        \item
          Add code to the implementation of each method code to store the 
          store the value of its argument(s) in a record placed at the end of the fluent-call-list. 
      \end{enumerate}
\end{description}
=======
\begin{enumerate}
  \item On the first step, we transform the grammar to a DPDA~$⟨Q,Γ⟩$ (LL(1) - linear grow, LR(1) - exponential blowup)
  \item For an arbitrary~$k$ the size of stack symbols turns to~$Γᵏ$.
  \item The number of states grow to~$Q⨉Γᵏ$.
  \item The number of stack symbols grow to~$Γᵏ⨉(Q²⨉Γ^{2k})^{Q⨉Γᵏ}$
\end{enumerate}
>>>>>>> 1832e4463687f81f73293ed44209fb0eb33834b6


\section{Intuition}
\label{section:recognizer}
%! TEX root = 00.tex
Formal presentation of the algorithm that compiles an LL(1) grammar into an implementation of a
  language recognizer with \Java generics
  is delayed to the next section.
This section gives an intuitive perspective on this algorithm:

We will first recall the essentials of the classical LL(1) parsing algorithm (\cref{section:essentials}).
Then, we will explain the limitations of the computational model
offered by \Java generics (\cref{section:limitations}).

The discussion proceeds to the observation
  that underlies our emulation of the parsing algorithm
  within these limitations.

Building on all these, \Cref{section:generation} can
  make the intuitive introduction to the main algorithm.
To this end, we revise the classical algorithm for converting
  an LL(1) grammar into an LL(1) parser, and explain
  how it is modified to generate a recognizer executable
  on the computational model of \Java generics.

\subsection{LL(1) Parsing}
\label{section:essentials}
An LL(1) parser is a DPDA allowed to peek at the next input terminal.
Thus,~$δ$, its transition function, takes two parameters: the current state of
the automaton, and a peek into the terminal next to be read.
The actual operation of the automaton
  is by executing in loop the step
  depicted in \cref{algorithm:ll-parser}.

\renewcommand\algorithmicdo{\textbf{\emph{}}}
\renewcommand\algorithmicthen{\textbf{\emph{}}}
\begin{algorithm}[p]
  \caption{\label{algorithm:ll-parser}
  DPDA algorithm for LL(1) parsing}
  \begin{algorithmic}
      \LET{$X$}{\Function pop()}\COMMENT{what's anticipated on input?}
      \IF[ancipated~$X$ is a terminal]{$X∈Σ$}
      \IF[read terminal is not anticipated~$X$]{$X≠\Function next()$}
      \STATE{\textsc{Reject}}\COMMENT{automaton halts in error}
      \ELSE[terminal just read was the anticipated~$X$]
      \STATE{\textsc{Continue}}\COMMENT{restart, popping a new~$X$, etc.}
        \FI
      \ELSIF[anticipating end-of-file]{$X=\$$}
        \IF[not the anticipated end-of-file]{$\$≠\Function next()$}
          \STATE{\textsc{Reject}}\COMMENT{automaton halts in error}
          \ELSE[terminal just read was the anticipated~$X$]
          \STATE{\textsc{Accept}}\COMMENT{all input successfully consumed}
        \FI
      \ELSE[anticipated terminal~$X$ must be a nonterminal]
      \LET{$R$}{$\Function δ(X, \Function peek())$}\COMMENT{which rule to reduce?}
      \IF[no rule found]{$R=⊥$}
          \STATE{\textsc{Reject}}\COMMENT{automaton halts in error}
        \ELSE[A rule was found]
        \LET{$(Z ::= Y₁,…,Yₖ)$}{R}\COMMENT{break~$R$ into left/right}
      \STATE{\textbf{assert}~$Z=X$}\COMMENT{$δ$ constructed to return valid~$R$ only}
          \STATE{$\Function print(R)$}\COMMENT{rule~$R$ has just been applied}
          \STATE{\textbf{For}~$i=k,…,1$,~$\Function push(Yᵢ)$}\COMMENT{push in reverse order}
      \STATE{\textsc{Continue}}\COMMENT{restart, popping a new~$X$, etc.}
        \FI % No rule found
      \FI % Main
\end{algorithmic}
  \vspace{0.3ex}
  \hrule
  \vspace{0.3ex}
  \scriptsize
  \begin{enumerate}
      \item
  Input is a stream of terminals drawn from alphabet~$Σ$, ending
  with a special symbol~$\$\ni Σ$.
      \item
  Stack symbols are drawn from~$Σ∪❴\$❵∪Ξ$, where~$Ξ$ is
the set of nonterminals of the grammar from which the automaton was generated.
\item
  Functions~$\Function pop()$ and~$\Function push(·)$
  operate on the push-down stack; function~$\Function next()$ returns
  and consumes the next terminal from the input stream;
  function~$\Function peek()$ returns this terminal without consuming it.
\item
The automaton starts with the stack with the start symbol~$S∈Ξ$ pushed
  into its stack.
  \end{enumerate}
\end{algorithm}

The DPDA maintains a stack of ‟anticipated” symbol, which may
  be of three kinds: a terminal drawn from the input alphabet~$Σ$,
  an -end-of=file symbol~$\$$, or,
  or one of~$Ξ$, the set of nonterminals of the underlying
  grammar.

If the state is an input terminal or the special, end-of-file,
  symbol~$\$$, then it must match
  the next terminal found in the input stream.
If no match is found, then the parser rejects.
The parser accepts if the input is exhausted with
  no rejections.

The more interesting case is that~$X$, the popped symbol
  is a nonterminal: the DPDA peeks into the next terminal in the input
  stream (without consuming it).
Based on this terminal, and~$X$ the transition function~$δ$
  determines~$R$ the derivation rule to use in order to derive~$X$.
The algorithm rejects if~$δ$ can offer no such rule.
Otherwise, it pushes into the stack, in reverse order, the symbols
  found in the right hand side of~$R$.

\subsection{LL(1) Parsing with \Java Generics?}
\label{section:limitations}
Can \cref{algorithm:ll-parser} be executed on the machinery
  available with \Java generics?
As it turns out, most operations conducted by the algorithm
  are easy.
The implementation of function~$δ$ can
  be found in the toolbox.
Similarly, any fixed sequence of push and pop
  operations on the stack can be conducted within a \Java
  transition function:
  the ‟\textbf{For}" at the algorithm can be unrolled.

Superficially, the algorithm appears to be doing a constant amount
  of work for each input symbol.
A little scrutiny falsifies such a conclusion.

In the case~$R=X→Y$,~$Y$
  being a nonterminal, the algorithm will conduct
  two~$\Function push(·)$ operations,
  one for~$X$ and one for~$Y$ before consuming a terminal from the input.
Further,~$δ$ may return next the rule~$Y→Z$,
  and then the rule~$Z→Z'$, etc.
Let~$k'$ be the number of such~$\Function push(·)$
  operations in a certain such incident.

Also, in the case~$k=0$ the DPDA does not push
  anything into the stack.
Further, in its next iteration, the DPDA conducts another
pop,
\[
  X'←\Function pop(),
\]
instruction and proceeds to consider this new~$X'$.
If it so happens, it could also be the case
  that the right hand side of rule~$R'$
  \[
    R' = δ(R', \Function peek())
  \]
  is once again empty,
  and then another~$\Function pop()$
    instruction occurs
\[
  X'←\Function pop(),
\]
  etc.
Let~$k^*$ be the number 
  of such instructions in a certain such mishap. 

  The cases $k' > 0$ and $k^*$ are not rarities.
For example, let us concentrate on the grammar 
  depicted in \cref{figure:issue-1}.
This grammar, inspired by the prototypical
  specification of \Pascal \urlref{http://www.fit.vutbr.cz/study/courses/APR/public/ebnf.html},
  shall serve as our running example.

The grammar preserves the ‟theme”
  of nested definitions of \Pascal,
  while trimming these down as much as possible.
The other major difference is that body of
routines must begin with~\cc{(} followed by \cc{begin}
    rather than a single ‟\cc{begin}”.

Some of the~$δ$ function values for this grammar are
\begin{equation}
  δ(n,t) = 3 
\end{equation}

\newsavebox{\Alphabet}
\begin{lrbox}{\Alphabet}
  \begin{tabularx}{0.40\linewidth}{l}
    \cc{program}, \cc{begin}, \cc{end},⏎
    \cc{label}, \cc{const}, \cc{id},⏎
    \cc{procedure},~\cc{;},~\cc{(}, \cc{()}
  \end{tabularx}
\end{lrbox}

\begin{figure}
  \caption{\label{figure:running}
    An LL(1) grammar over the alphabet
    \[
      Σ = \left❴\usebox\Alphabet\right❵.
    \]
    (inspired by the original \Pascal grammar; to serve as
    our running example)
  }
  \begin{Grammar}
    \begin{aligned}
      \<Program> & ::= \cc{program} \cc{id} \<Parameters>~\cc{;} \<Definitions>\~\<Body>\hfill⏎
      \<Body> & ::= \cc{begin} \cc{end}\hfill⏎
      \<Definitions> & ::= \<Labels>\~\<Constants>\~\<Nested>\hfill⏎
      \<Labels> & ::= ε \| \cc{label} \<Label>\~\<MoreLabels> \hfill⏎
      \<Constants> & ::= ε \| \cc{const} \<Constant>\~\<MoreConstants> \hfill⏎
      \<Label> & ::=\cc{;} \hfill⏎
      \<Constant> & ::=\cc{;} \hfill⏎
      \<MoreLabels> & ::= ε \| \<Label>\~\<MoreLabels>\hfill⏎
      \<MoreConstants> & ::= ε \| \<Constant>\~\<MoreConstants>\hfill⏎
      \<Nested> & ::= ε \| \<Procedure>\~\<Nested> \hfill⏎
      \<Procedure> & ::= \cc{procedure} \cc{id} \<Parameters>~\cc{;} \<Definitions>~\cc{(} \<Body> \hfill⏎
      \<Parameters> & ::= ε \| \cc{()} \hfill⏎
    \end{aligned}
  \end{Grammar}
\end{figure}



\subsection{Generation of LL(1) DPDA Parsers}
\label{section:generation}
% How can we encode it?
The corresponding \Java fluent API ought to have equivalent features.
The input string - since we transform each
  terminal to a method invocation, it is clear that the rest of the input is
  simply the rest of the method chain.
The current state of the parsing is transformed into a \Java type
  (\kk{class} or \kk{interface}) ; this way we can control for each state
  which operation can be made by determining the method of that type.
The environment is realized by type arguments of the states' \Java types.

Using these transformers, we can describe a state of the parser,
  with different stacks using a single type.
This is of course necessary in order to describe all posible configurations
  in a finite number of \Java types.

%
\subsection{LL states}
A state during the parser's computation is a grammar rule~$r = A→αβ$, and an
index in it, i.e.,~$A→α·β$.
It means that the parser is currently processing rule~$r$, it already realized parsed string~$alpha$
  from the input, and it expect to parse the string~$β$ next.
The number of states then, produced by \Fajita for a grammar is linear in the grammar size.

\subsection{Environment}

\begin{algorithm}[p]
  \caption{\label{algorithm:llclosure}
  function~$\Function closure(a,b)$: generates a closure of action from the original ll algorithm}
  \begin{algorithmic}
    \INPUT{a nonterminal~$a$}
    \INPUT{a terminal~$b$}
    \OUTPUT{the closure of consecutive actions}
    \LET{$L$}{$[]$}
    \LET{$x$}{$a$}
    \IF{$∀x→α. b ∉ \Function first(α)$}
      \RETURN{\textrm{reject}}
    \FI
    \WHILE{$\textrm{true}$}
      \IF{$x∈ξ$}
        \STATE{let~$y₁ y₂… yₖ$ be~$α$ s.t.~$∃x→α∈g∧b∈\Function first(α)$}
        \STATE{$\Function append(l,yₖ,y_{k-1},…,y₂)$}
        \STATE{$x=y₁$}
      \ELSE[$x$ must be~$b$]
        \BREAK
      \FI
    \DONE
    \RETURN{$L$}
  \end{algorithmic}
\end{algorithm}


\section{Theoretical background}
\label{section:theoretical-background}
\input background

\section{Bootstrapping \Fajita}
\label{section:bootstrapping}
\input bootstrapping

\section{Conclusion and Future Work}
\label{section:zz}
The theoretical part of this work dwells on the proof of~\cref{Theorem:Gil:Levy}.
Its engineering part is concerned is
  a software implementation that would make JAVA
  such as in~\cref{Figure:fluent} generate
  the required \Java JAVA that realizes the
  defined grammar, so that JAVA such as
  found in~\cref{Figure:toilette:legal} is type-correct,
  whereas JAVA found in~\cref{Figure:toilette:illegal} is not.
†{%
An important side effect of the implementation is that IDEs with built-in JAVA
completion
 (found e.g., in Eclipse~†{\href{http://www.eclipse.org/}{Eclipse home page}} and IntelliJ~†{\href{https://www.jetbrains.com/idea/}{Intellij home page}})
 will assist the programmer in making a correct use of the API.
 }

Accordingly, the contribution of this work is double folded:
  gaining better understanding of the computational expressiveness of
  \Java generics and type hierarchy, and, a better tool
  for designing, experimenting with and perfecting fluent APIs.

How about EBNF\@? Reference perhaps to Anna Beckermans thesis.\cite{Tomer:also try to trace citations from wikipedia}
It can be done. Further research directions might be found in exploring this venue. 


\bibliographystyle{abbrv}\small
\bibliography{author-names,other-shorthands-abbreviated,%
 publishers-abbreviated,%
 conferences-abbreviated,%
 journals-abbreviated,journals-full,%
 yogi-book,yogi-practice,yogi-journal,yogi-tr,%
 GPCE,OOPSLA,PLDI,USENIX,ECOOP,%
 00,yogi-confs}

\clearpage
\appendix
\section{The JLR Recognizer}
\input lr-algorithm

\end{document}
