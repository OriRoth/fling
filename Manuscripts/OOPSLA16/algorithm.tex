%! TEX root = 00.tex

\subsection{Solving~$k'$ using closure}
\subsection{Solving~$k^*$}
   
\subsection{Main Algorithm}
The idea of the main is simple


\endinput

In~\cref{section:intuition} we presented the practical issues of
  implementing the LL(1) parser ‟as is”, using two indexes:
  the substitution factor~$k'$, and the pop factor~$k^*$.

This section will cover the solutions to these issues and
  introduce the algorithm for the recognizer generation.

\subsection{Substitution Factor~$k'$}
The problem caused by the substitution factor if~$k'>0$
  is that the recognizer needs to perform several substitution
  of the top of the stack with the next rule to parse.

What we would like to have, is a single substitution that composes
  all consecutive substitutions at one step.

An important observation is that the only parameter
  that changes during runtime is the top of the stack.
Thus, if we solve this issue for every nonterminal,
  the sequence of operations could be computed statically.

The computation process is rather simple:
  given a nonterminal~$A∈Ξ$ (the top of the stack)
  †{in case there's a terminal at the top of the stack~$k'=0$
  and there are no substitutions}
  and the next input symbol~$t∈Σ∪❴\$❵$, the algorithm iteratively
  builds the RHS of the rule which is the ‟closure” of the consecutive rules.

The algorithm applies the following inductive step:
  use~$r = \Function predict(A,t)$ to predict the next rule as the LL(1)
  parser would ; add the RHS of~$r$ to the beginning of the result.
If the first symbol of the result is~$t$, finish.
Otherwise, it is a nonterminal. Let set~$A$ to be that nonterminal, and
  apply the inductive step again.
The full algorithm is given in~\cref{algorithm:llclosure}.

\cref{algorithm:llclosure} presents
\begin{algorithm}[p]
  \caption{\label{algorithm:llclosure}
  An algorithm for computing the rule~$\Function closure(r,b)$ for some rule~$r∈P$ and
  terminal~$b∈Σ$, when~$Ξ$ is the nonterminals set.}
  \begin{algorithmic}
    \STATE{$(A::=Y₀…Yₖ)←r$} \COMMENT{break~$r$ into its LHS & RHS}
    \IF[$r$ cannot be derived to begin with~$b$]{$b∉\Function First(Y₀…Yₖ)$}
      \STATE{\textsc{Error}}
    \FI
    \STATE{$\text{LHS}=A$} \COMMENT{the output rule shares the LHS with the input}
    \STATE{$\text{RHS}=Y₀…Yₖ$} \COMMENT{initially, the RHS is the original rule}
    \LET{$h$}{$\Function PopHead(\text{RHS})$} \COMMENT{unroll the first symbol in RHS}
    \WHILE[$h$ is always a nonterminal in the loop]{$h∉Σ$}
      \FOR[for each~$h$-rule]{$h::=Y₀'…Yₖ'←get(P)$}
        \IF[if this rule matches]{$b∈\Function First(Y₀'…Yₖ')$}
          \STATE{$\text{RHS}=Y₀'…Yₖ' + \text{RHS}$} \COMMENT{add it to closure}
          \BREAK \COMMENT{since were in LL(1), can't be other}
        \ENDIF
      \ENDFOR
      \STATE{$h←\Function PopHead(\text{RHS})$} \COMMENT{unroll the first symbol in RHS}
    \ENDWHILE
    \RETURN{$\text{LHS}::=\text{RHS}$}
  \end{algorithmic}
\end{algorithm}


