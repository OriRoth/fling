  
  
  
  Also, semantics is typically one of two.
\emph{Eager} execution semantics is
found e.g., in \Java's \cc{StringBuilder}\urlref{https://docs.oracle.com/javase/7/docs/api/java/lang/StringBuilder.html}

\begin{quote}
  \parbox[c]{50ex}{\javaInput[]{../Fragments/stringbuilder-example.fragment}}⏎
\end{quote}

In this semantics, each method in the chain executes a computation,
  whose result, is made the receiver of the next method
  in the chain.

Conversely, \emph{delayed} execution semantics, occurring e.g., in jOOQ,
  maintain a ‟production line”,
  by which each method in the fluent call chain receives a
  ‟action builder” (called ‟query builder” in jOOQ and LINQ)
  data record from the method preceding it, adds a bit of
  information to this record, and passes it to the next method
  in the chain.
If an action builder is sufficiently defined it can
  produce an action record.
And, action records, may execute
  when specifically requested to so.


