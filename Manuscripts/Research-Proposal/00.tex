\title{\Huge \SELF---Fluent API for \textsc{Java} with Bottom-Up and Top-Down Parsing} 
\documentclass[10pt,twocolumn]{article}

\usepackage{\jobname}

\author{Tome Levy\\
	Department of Computer Science\\
	Technion---Israel Institute of Technology\\
	\texttt{\small \href{mailto:stlevy@campus.technion.ac.il}{stlevy@campus.technion.ac.il}}}
\date{\small Advisor: Prof.\ Yossi Gil}

\begin{document}
\maketitle
  
\begin{abstract}
	The term ‟Fluent API” has become a buzzword. So has the related term
‟DSL”. The objective of this research is to use the power of \Java generics
to combine the two. Specifically, to create a software system, to be called
\Self, that would make the definition of a DSL for fluent API as easy as the
writing an EBNF for this DSL\@. And, just like an ouroboros, the definition of this
EBNF will be through a fluent API generated by \Self itself.
 
\end{abstract}

\section{Introduction}
\section{Fluent APIs Java}
Ever after their inception\urlref{http://martinfowler.com/bliki/FluentInterface.html} \emph{fluent APIs}
  increasingly gain popularity~\cite{Hibernate:06,Freeman:Pryce:06,Larsen:2012} and research
  interest~\cite{Deursen:2000,Kabanov:2008}.
In many ways, fluent APIs are a kind of
  \emph{internal} \emph{\textbf Domain \textbf Specific \textbf Languages}~\cite{VanDeursen:Klint:2000,Hudak:1997,Fowler:2010}:
They make it possible to enrich a host programming language without changing it.
Advantages are many: base language tools (compiler, debugger, IDE, etc.) remain
  applicable, programmers are saved the trouble of learning a new syntax, etc.
However, these advantages come at the cost of expressive power;
  in the words of Fowler:
  ‟\emph{Internal DSLs are limited by the syntax and structure of your base language.}”†
  {M. Fowler, \emph{Language Workbenches: The Killer-App for Domain Specific Languages?},
    2005
    \newline
  \url{http://www.martinfowler.com/articles/languageWorkbench.html#InternalDsl}}.
Indeed, in languages such as \CC, fluent APIs
  often make extensive use of operator overloading (examine, e.g., \textsf{Ara-Rat}~\cite{Gil:Lenz:07}),
  but this capability is not available in \Java.

Despite this limitation, fluent APIs in \Java can be rich, expressive,
and in many cases cab boost productivity in elegant, neat code snippets such as

\begin{quote}
  \javaInput[minipage,width=\linewidth,left=-2ex]{../Fragments/camel-apache.java.fragment}
\end{quote}

(a use case of Apache Camel~\cite{Ibsen:Anstey:10}, open-source integration
framework), and,

\begin{quote}
  \javaInput[minipage,width=\linewidth,left=-2ex]{../Fragments/jOOQ.java.fragment}
\end{quote}

(a use case of jOOQ\urlref{http://www.jooq.org}, a framework for writing SQL
like code in \Java, much like LINQ project~\cite{Meijer:Beckman:Bierman:06} in
the context of \CSharp).

Other examples of fluent APIs in \Java are abundant:
jMock~\cite{Freeman:Pryce:06},
Hamcrest\urlref{http://hamcrest.org/JavaHamcrest/},
EasyMock\urlref{http://easymock.org/},
jOOR\urlref{https://github.com/jOOQ/jOOR},
jRTF\urlref{https://github.com/ullenboom/jrtf}
and many more.

The actual implementation of these many examples is traditionally not carried
out in the neat manner it could possibly take. Our reason for saying this is
that the fundamental problem in fluent API design is the decision on the
‟language”. This language is the precise definition of which sequences of
method applications are legal and which are not.

As it turns out, the question of whether a BNF definition of such a language
can be ‟compiled” automatically into a fluent API implementation is, a
question of the computational power of the underlying language. In \Java, the problem
is particularly interesting since in \Java (unlike e.g., \CC~\cite{Gutterman:2003}),
the type system is (probably) not Turing complete.

The question is then, what can be computed, and what can not be computed by
coercing the type system and the type checker of a certain programming language
to do abstract computations it was never meant to carry out? And, why should we
care?

The previous jOOQ example suggests that jOOQ imitates SQL, but, is it possible at
all to produce a fluent API for the entire SQL language, or XPath, HTML,
regular expressions, BNFs, EBNFs, etc.?

Of course, with no operator overloading it is impossible
to fully emulate tokens; method names though make a good substitute for tokens, as done
in
\begin{quote}
  \javaInput[minipage,width=45ex,left=-2ex]{../Fragments/jOOQ-mini.java.fragment}.
\end{quote}

The questions that motivate this research are:
\begin{quote}
  \begin{itemize}
    \item Given a BNF specification of a DSL, determine whether there exists a
      fluent API in \Java that can be made for this specification?

    \item In the cases that such fluent API is possible, can it be produced
      automatically?

    \item Is it feasible to produce a \emph{compiler-compiler} such as
      Bison~\cite{Bison:manual} to convert such language specification into a
      fluent API?

\end{itemize}
\end{quote}

Inspired by the theory of formal languages and automata,
  this study explores what can be done with fluent APIs in \Java.

\section{A Brief Review of Fajita - our Main Goal}
Consider \Cref{figure:sql-bnf}, a BNF specification of a fluent API for a certain
fragment of SQL.

\begin{figure}[H]
  \caption{\label{figure:sql-bnf}
    A BNF for a fragment of SQL select queries.
  }
  \begin{Grammar}
    \begin{aligned}
      \<Query> & \Derives \cc{select()} \<Quant>\~\cc{from(Table.\kk{class}\!\!\!)} \<Where> \hfill⏎
      \<Quant> & \Derives \cc{all()} \hfill⏎
                  & \| \cc{columns(Column[].\kk{class}\!\!\!)} \hfill⏎
      \<Where> & \Derives \cc{where()} \cc{column(Column.\kk{class}\!\!\!)} \<Operator> \hfill⏎
                  & \|ε \hfill⏎
      \<Operator> & \Derives \cc{equals(Expr.\kk{class}\!\!\!)}\hfill⏎
                  & \| \cc{greaterThan(Expr.\kk{class}\!\!\!)} \hfill⏎
                  & \|\cc{lowerThan(Expr.\kk{class}\!\!\!)} \hfill
    \end{aligned}
  \end{Grammar}
\end{figure}

To create a \Java implementation that realizes this fluent API,
  the designer feeds the grammar to \Fajita, as in
  \cref{figure:sql-bnf-java}.

\begin{figure}[H]
  \caption{\label{figure:sql-bnf-java}
    A \texorpdfstring{\Java}{Java} code excerpt defining the BNF specification of the fragment SQL
    language defined in \cref{figure:sql-bnf}.}
  \javaInput[minipage,width=\linewidth,left=-6ex]{sql.bnf.listing}
\end{figure}

We see that \Fajita's API is fluent in itself, and the
  call chain in \cref{figure:sql-bnf-java}, is structured almost
  exactly as in derivation rules in \cref{figure:sql-bnf}.
In particular, the code in \cref{figure:sql-bnf-java} shows how fluent API specification in \Fajita
  may include parameterless methods (\cc{select()}, \cc{all()} and \cc{where}) as well as methods which
  take parameters, e.g., method \cc{column} taking parameter of type \cc{Column} and
    method \cc{from} taking a \cc{Table} parameter.

Other than the derivation rules, \Fajita needs to be told the start rule
  and the sets of terminals and nonterminals.
These are specified in the first method call in the chain where,
  the enumerate types \cc{SQLTerminals} and \cc{SQLNonTerminals} are:

\begin{quote}
  \javaInput[minipage,width=\linewidth,left=-4ex]{sql.enums.listing}
\end{quote}

The call \cc{.go()}, occurring last in the chain, makes \Fajita generate types
and methods realizing the fluent API, in such a way that legal use of the API
like in \cref{figure:sql:legal} is syntactically legal,

\begin{figure}[H]
  \caption{\label{figure:sql:legal}
  Legal sequences of calls in the sql fragment example}
  \javaInput[minipage,width=\linewidth,left=-4ex]{sql.legal.listing}

\end{figure}

while snippets disobeying the BNF specification in \cref{figure:sql-bnf} like
\cref{figure:sql:illegal} , do not type check.

\begin{figure}[H]
  \caption{\label{figure:sql:illegal}
  Illegal sequences of calls in the sql fragment example}
  \javaInput[minipage,width=\linewidth,left=-4ex]{sql.illegal.listing}
\end{figure}

\section{The many trials of doing it in Java}
Fluent APIs are neat, but design is complicated,
involving theory of automata, type theory, language design, etc.

And still, automatic generation of these exist to some extent.
One example, is fluflu\urlref{https://github.com/verhas/fluflu}: a software artifact that uses
\Java annotations to define deterministic finite automata(henceforth, DFA), and then
compiles it to a fluent API based on this automaton. Usage example of fluflu is
depicted at \cref{figure:fluflu}.

\begin{figure}[ht]
  \caption{\label{figure:fluflu}
    Usage example of fluflu, generation a fluent
  API for the regular expression~\texorpdfstring{$a^*$}{a*}}
  \javaInput[minipage,left=-4ex]{fluflu.example.listing}
\end{figure}
The code excerpt in~\cref{figure:fluflu} defines a single stated DFA using
fluflu. The state is both initial and accepting, with a single self
transition, labeled with terminal~$a$. The regular language realized by the
automaton is~$L=a^*$.

Although fluent API generation was ‟achieved”, this example has many flaws, some of them are:

\begin{enumerate}
  \item The usage is complicated and the resulted code is messy.
  \item Defining a DFA is harder then writing, for example, the equivalent regular
        expression (and in some cases, the size of the DFA might be exponentially bigger)
  \item Languages defined by DFA can only be regular, a rather small
        class of languages.
\end{enumerate}

A question rises after seeing \Fajita and fluflu: how come the representation
of simple languages such as regular languages is so complex (as seen in fluflu)
and the representation of a more complex class of languages, is rather simple?
As it turns out, two factors are involved in the answer.

The first is the complexity of \emph{representing the language}.
Regular language are mostly defined by regular expression, but the language of
all regular expressions is actually context-free and not regular!
(Intuitively, since regular expressions uses parenthesizing). On the contrary,
the definition of context-free languages is usually done with a BNF (or a
context-free grammar), and the language of all BNFs is regular!

Thus, in order to define a fluent API for generating regular languages, one needs a
fluent API defined by a context-free language, while in order to define a fluent API for
context-free languages, only a regular fluent API is needed.

Since fluflu ‟knows” only how to generate fluent APIs for regular languages,
it cannot represent a fluent API for the generation of itself.
Respectively, since \Fajita's language is regular, and can generate context-free
  fluent APIs, \Fajita can generate itself.

The second factor is the \emph{practical complication} of generating the \Java types to
support the fluent API. As covered in \cref{section:example} the generation of
fluent APIs for regular languages is rather simple, while the generation of
those for context-free languages is the main challenge of this thesis.
\section{Contribution}
The main results of this research are:
\begin{quote}
  \begin{enumerate}
  \item If the DSL specification is that of a deterministic context-free
    language, then a fluent API exists for the language, but we do not know
    whether such a fluent API exists for more general languages.
  \par
  Recall that there are universal cubic time parsing
  algorithms~\cite{Cocke:1969,Earley:1970,Younger:1967} which can parse (and recognize) any
  context-free language. What we do not know is whether algorithms of this sort
  can be encoded within the framework of the \Java type system.
  \item
  There exists an algorithm to generate a fluent API that realizes any
  deterministic context-free languages. Moreover, this fluent API can create
  at runtime, a parse tree for the given language. This parse tree can then be
  supplied as input to the library that implements the language's semantics.
    \item
  Unfortunately, a general purpose compiler-compiler
  is not yet feasible with the current algorithm.
  \begin{itemize}
    \item One difficulty is usual in the fields of formal languages:
      The algorithm is complicated and relies on
      modules implementing complicated theoretical results, which, to the best of our
      knowledge, have never been implemented.
    \item Another difficulty is that a certain design decision in the
      implementation of the standard \texttt{javac} compiler is likely to make it choke on the
      \Java code generated by the algorithm.
  \end{itemize}
  \item
    We did implement a prototype for a compiler-compiler that works for LL(1) grammars.
  \end{enumerate}
\end{quote}

Other concrete contributions made by this work include
\begin{itemize}
  \item the understanding that the definition of fluent APIs is analogous to
      the definition of a formal language.
  \item a lower bound (deterministic pushdown automata)
    on the theoretical ‟computational complexity” of the \Java type system.
  \item an algorithm for producing a fluent API for deterministic context-free languages (even if impractical).
  \item a collection of generic programming techniques, developed towards this algorithm.
  \item a demonstration that the runtime of Oracle's \texttt{javac} compiler may be exponential in the program size.
  \item a new interpretation to the LL parsing algorithm, under a ‟realtime” constraint.
\end{itemize}

The theoretical result that any deterministic context-free grammar can be
automatically ‟compiled” to fluent API takes a toll of exponential blowup.
Specifically, the construction first builds a deterministic pushdown automaton
whose size parameters,~$g$ (number of stack symbols), and,~$q$ (number of
internal states), are polynomial in the size of the input grammar. This
automaton is then emulated by a weaker automaton, with as many as
\[
  O\left(g^{O(1)}\left(q²g^{O(1)}\right)^{qg^{O(1)}}\right)
\]
stack symbols.
This weaker automaton is then ‟compiled” into a collection of generic \Java types,
where there is at least one type for each of these symbols.


This work present an algorithm to compile an LL grammar of a fluent API
language into a \Java implementation whose size parameters are linear in
the size parameters of the LL parser generated by the classical
algorithm (\cref{algorithm:generation}) for computing LL parsers,
i.e., the performance loss due to implementation within the \Java
type checker is as small as we can hope it to be.

The savings are made possible by the use of a stronger automaton (the \RLLp,
described in detail below in \cref{section:realtime}) for the emulation, and
more efficient ‟compilation” of the \RLLp into the \Java type system.
We also present \Fajita
†{\itshape \textbf Fluent \textbf API for \textbf J\textsc{ava}
  (\textbf Inspired by the \textbf Theory of \textbf Automata)
}
a \Java tool that implements this algorithm.

\section{Related Work}

It has long been known
  that \CC templates are Turing complete in the following precise sense:

\begin{Proposition}
  \label{Theorem:Gutterman}
  For every Turing machine,~$m$, there exists a \CC program,~$Cₘ$ such that
    compilation of~$Cₘ$ terminates if and only if
      Turing-machine~$m$ halts.
      Furthermore, program~$Cₘ$ can be effectively generated from~$m$~\cite{Gutterman:2003}.
\end{Proposition}

Intuitively, this is due to the fact that templates in \CC
  feature both recursive invocation and conditionals (in the form of
  ‟\emph{template specialization}”).

In the same fashion, it should be mundane to make the judgment that
  \Java's generics are not Turing-complete since they offer no conditionals.
Still, even though there are time complexity results regarding type systems in functional
  languages, we failed to find similar claims for \Java.

Specialization, conditionals, \kk{typedef}s and other features of \CC templates,
  gave rise to many advancements in template/generic/generative programming
  in the language~\cite{Austern:98,Musser:Stepanov:1989,Backhouse:Jansson:1999,Dehnert:Stepanov:2000},
  including e.g., applications in numeric libraries~\cite{Veldhuizen:95,Vandevoorde:Josuttis:02},
  symbolic derivation~\cite{Gil:Gutterman:98}
  and a full blown template library~\cite{Abrahams:Gurtovoy:04}.

Garcia et al.~\cite{Garcia:Jarvi:Lumsdaine:Siek:Willcock:03} compared
  the expressive power of generics in half a dozen major programming languages.
  In several ways, the \Java approach~\cite{Bracha:Odersky:Stoutamire:Wadler:98}
  did not rank as well as others.

Not surprisingly, work on meta-programming using \Java generics,
  research concentrating on other means for enriching the language,
  most importantly annotations~\cite{Papi:08}.

The work on SugarJ~\cite{Erdweg:2011} is only one of many other attempts
  to achieve the embedded DSL effect of fluent APIs by language extensions.

Suggestions for semi-automatic generation can be found in the work of
Bodden~\cite{Bodden:14} and on numerous locations in the web.  None of these
materialized into an algorithm or analysis of complexity.  However, there is a
software artifact (the previously reviewed
fluflu\urlref{https://github.com/verhas/fluflu}) that automatically generates a
fluent API that obeys the transitions of a given finite automaton.

The challenges of \Java generic programming were highlighted by Garcia et
al.~\cite{Garcia:Jarvi:Lumsdaine:Siek:Willcock:03} research on the expressive
power of generics in half a dozen major programming languages,
Indeed, unlike \CC~\cite{Austern:98,Musser:Stepanov:1989,
Backhouse:Jansson:1999, Dehnert:Stepanov:2000,Gil:Gutterman:98,Abrahams:Gurtovoy:04}, the literature on meta-programming with \Java
generics is minimal.


\section{Terminology}
\input{terminology}

\section{This proposal}
\label{Section:proposal}
The thesis propounded by this research is that API design, and especially fluent API design
  can and should be made in terms of language design.
Software missionaries and preachers such as Fowler~\cite{Fowler:2005} have long claimed
  that API design resembles the design of a \textbf Domain \textbf Specific \textbf Language
  (henceforth \emph{DSL}, see, e.g.,~\cite{VanDeursen:Klint:2000,Hudak:1997,Fowler:2010} for review articles).
   In the words of Fowler ``The difference between API design and DSL design is then rather small''~\cite{Fowler:2005})

The objective of this research is
  to take the unification of the notions of DSL and (fluent) API
  design one step further in automating the creation of fluent API out
  of a DSL specification.

The basic idea is that the programmer specifies a fluent API, 
  and, then, this specification is then automatically translated 
  to an implementation of a fluent API that conforms 
  with this specification.
This translation generates the intricate type hierarchy 
  and methods of types in it in such a way 
  that only sequence of calls that conform
  to the specification would 
  compile correctly (concretely, type-check).
 

  To illustrate, consider the toilette seat example. 
In this example, 
  there are a total of six methods that might be invoked. 
\begin{quote}	
  \begin{tabular}{lll}
    \cc{male()}   & \cc{raise()} & \cc{urinate()}⏎
    \cc{female()} & \cc{lower()} & \cc{defecate()}⏎
  \end{tabular}
\end{quote}
A fluent API design specifies the order in which such calls can be made.

The \emph{first} novelty in this research is that the fluent API definition is 
  through a CFG, written as a BNF. 
\cref{Figure:BNF} is such a specification for the toilette seat problem. 
\SELF takes this as input, and in responses generates the corresponding 
  \Java type hierarchy. 

\begin{figure}[htbp]
  \begin{equation*}
    \def\<#1>{\/⟨\/\text{\textit{#1}}\/⟩\/{ }}
    \def\|{~~|~~~}
    \let\oldCc=\cc
    \def\cc#1{{\footnotesize\oldCc{#1}}{\ }}
    \small
    \begin{aligned}
      \<Visitors>         & ::=  \<Down-Visitors>     \hfill⏎
      \<Down-Visitors>    & ::=  \<Down-Visitor>      \<Down-Visitors>  \hfill⏎
      {}                  & \|   \<Raising-Visitor>   \<Up-Visitors>    \hfill⏎
      {}                  & \|   ε                    \hfill⏎
      \<Up-Visitors>      & ::=  \<Up-Visitor>        \<Up-Visitors>    \hfill⏎
      {}                  & \|   \<Lowering-Visitor>  \<Down-Visitors>  \hfill⏎
      {}                  & \|   ε                    \hfill⏎
      \<Up-Visitor>       & ::=  \cc{male()}          \cc{urinate()}    \hfill⏎
      \<Down-Visitor>     & ::=  \cc{female()}        \<Action>         \hfill⏎
                          & \|                  \cc{male()}          \cc{defecate()}  \hfill⏎
      \<Raising-Visitor>  & ::=  \cc{male()}          \cc{raise()}      \cc{urinate()}  \hfill⏎
      \<Lowering-Visitor> & ::=  \cc{female()}        \cc{lower()}      \<Action>       \hfill⏎
                                & \|                  \cc{male()}          \cc{lower()}           \cc{defecate()}  \hfill⏎
      \<Activity>         & ::=  \cc{urinate()}       \hfill⏎
                          & \|                  \cc{defecate()}  \hfill⏎
    \end{aligned}
  \end{equation*}
  \caption{A BNF grammar for the toilette seat problem}
  \label{Figure:BNF}
\end{figure}

A second novelty of \SELF is that the specification of a BNF such as the provided in
  \cref{Figure:BNF} can be made in using a \Java fluent API.
To do so, it is first necessary to  
   define the set of terminals
  \begin{lcode}{Java}
enum ToiletteTerminals implements Terminal {
  male, female, 
  urinate, defecate, 
  lower, raise;
}
\end{lcode}
  and then the set of grammar symbols 
  \begin{lcode}{Java}
enum ToiletterSymbols implements Symbol {
  Visitors, Down_Visitors, Up_Visitors, Up_Visitor, Down_Visitor, 
  Lowering_Visitor, Raising_Visitor,
  Actitivity
};
  \end{lcode}
Once these two are made, the grammar can be defined,
  using a fluent API generated by \SELF itself as shown
  in \cref{Figure:fluent}. 

\begin{figure}[htbp]
  \begin{lcode}{Java}
BNF toiletteBnf = new BNFBuilder()
  .with(ToiletterTerminals.class)
  .with(ToiletterSymbols.class)
  .start(Visitors)
  .derive(Visitors)
    .to(Down_Visitors)
  .derive(Down_Visitors)
    .to(Down_Visitor).and(Down_Visitors)
    .or().to(Raising_Visitor).and(Up_Visitors)
    .or().toEpsilon()
  .derive(Up_Visitors)
    .to(Up_Visitor).and(Up_Visitors)
    .or().to(Lowering_Visitor).and(Down_Visitors)
    .or().toEpsilon()
  .derive(Up_Visitor)
    .to(male).and(urinate)
  .derive(Down_Visitor)
      .to(female).and(Action)
      .or().to(male).and(defecate)
  .derive(Raising_Visitor)
      .to(male).and(raise).and(urinate)
  .derive(Lowering_Visitor)
    .to(female).and(lower).and(Action)
    .or().to(male).and(lower).and(defecate)
  .derive(Activity).to(urinate)
    .or().to(defecate)
  .finish();
  \end{lcode}
  \caption{A BNF grammar for the toilette seat problem}
  \label{Figure:fluent}
\end{figure}


Processing programming languages
\begin{description}
  \item[Lexical analysis] - the first step of the process in which the character strings generated by the 
  programmer are aggregated to the abstract tokens defined by the language designer.
  \item[Syntactical analysis (parsing) ] - the second step, in which the processed strings of tokens 
  conform to the rules of a formal grammar defined by the language's BNF (or EBNF).
  \item[Semantical analysis] - the next step, usually performed in unison with the previous step, 
  in which the legal token sequences are given their semantic meaning.
\end{description}
Specifically, the proposal is that API design of follows the footsteps of
Accordingly, the designer of a fluent API has to follow these three conceptual
steps.
First is the identification of the \emph{vocabulary}, i.e.,
the set of method calls including type arguments that may take part in the
fluent API.
In this fluent API example
\begin{lcode}{Java}
allowing (any(Object.class))
  ¢¢.method("get.*")
  ¢¢.withNoArguments();
\end{lcode}
then, there are three method calls, and the vocabulary has three items in it.
\begin{itemize}
  \item~$ℓ₁ = \cc{any(Class<?>)}$
  \item~$ℓ₂ = \cc{allowing($ℓ₁$)}$
  \item~$ℓ₃ = \cc{method(String)}$
  \item~$ℓ₄ = \cc{withNoArguments()}$
\end{itemize}


\section{Bootstrapping}
\label{Section:zz}
The theoretical part of this work dwells on the proof of~\cref{Theorem:Gil:Levy}.
Its engineering part is concerned is
  a software implementation that would make JAVA
  such as in~\cref{Figure:fluent} generate
  the required \Java JAVA that realizes the
  defined grammar, so that JAVA such as
  found in~\cref{Figure:toilette:legal} is type-correct,
  whereas JAVA found in~\cref{Figure:toilette:illegal} is not.
†{%
An important side effect of the implementation is that IDEs with built-in JAVA
completion
 (found e.g., in Eclipse~†{\href{http://www.eclipse.org/}{Eclipse home page}} and IntelliJ~†{\href{https://www.jetbrains.com/idea/}{Intellij home page}})
 will assist the programmer in making a correct use of the API.
 }

Accordingly, the contribution of this work is double folded:
  gaining better understanding of the computational expressiveness of
  \Java generics and type hierarchy, and, a better tool
  for designing, experimenting with and perfecting fluent APIs.

How about EBNF\@? Reference perhaps to Anna Beckermans thesis.\cite{Tomer:also try to trace citations from wikipedia}
It can be done. Further research directions might be found in exploring this venue. 


\bibliographystyle{abbrv}
%\bibliography{publishers,other_shorthands,institutions,author_names,journals_full,yogi-journal,yogi-book,00}
\bibliography{author-names,other-shorthands,publishers-abbreviated,yogi-book,00}
\end{document}

Processing programming languages
\begin{description}
  \item[Lexical analysis] - the first step of the process in which the character strings generated by the 
  programmer are aggregated to the abstract tokens defined by the language designer.
  \item[Syntactical analysis (parsing) ] - the second step, in which the processed strings of tokens 
  conform to the rules of a formal grammar defined by the language's BNF (or EBNF).
  \item[Semantical analysis] - the next step, usually performed in unison with the previous step, 
  in which the legal token sequences are given their semantic meaning.
\end{description}
Specifically, the proposal is that API design of follows the footsteps of
Accordingly, the designer of a fluent API has to follow these three conceptual
steps.
First is the identification of the \emph{vocabulary}, i.e.,
the set of method calls including type arguments that may take part in the
fluent API.
In this fluent API example
\begin{lcode}{Java}
allowing (any(Object.class))
  ¢¢.method("get.*")
  ¢¢.withNoArguments();
\end{lcode}
then, there are three method calls, and the vocabulary has three items in it.
\begin{itemize}
  \item~$ℓ₁ = \cc{any(Class<?>)}$
  \item~$ℓ₂ = \cc{allowing($ℓ₁$)}$
  \item~$ℓ₃ = \cc{method(String)}$
  \item~$ℓ₄ = \cc{withNoArguments()}$
\end{itemize}
