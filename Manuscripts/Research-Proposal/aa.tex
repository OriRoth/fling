For the last few years I have assisted my father in his CNC workshop, finalizing
and burnishing the metallic components he would produce. Since my father's machines
would hardly ever produce a flawless component on their own, additional time had 
to be put into cutting away the excess and flaws put into the products.	
When dealing with exceptionally large materials such as long rods of steel, the probability
of deforming the product increases linearly, so that working on a short stub would
usually be far less dangerous than its long counterpart. When dealing with large 
scale software, I expected to encounter a similar behavior; having each piece of code
contributing a small and equal share to the chances of weaving an error into the software.
It has been discovered by previous papers, such as \emph{Eli's thesis} that this assumption
is not entirely correct.
Imagine a metallic rod held at one side. The longer the rod, the higher the leverage at it's 
far side, which makes it subject more strongly to centrifugal force or the effects of gravity.
In such a case, by knowing the properties of the metal, we can calculate the exact length 
which will cause the rod to snap. But imagine now, that the longer the rod, the stronger it gets;
even without any changes to it's matter. As if a greater mass of it's substence made the entire
object more durable.
We found out that code behaves in a similar way; meaning, the longer a method is, the less 
likely it is for any part of it to be flawed. This understanding brings about some important
questions. Such as, at what size would this metaphorical rod achieve it's peak durability?
Can a mathematical formula be used to predict it's characteristics at a given length?
And how would different elements and alloys (or in a more concrete sense, different programming
languages) follow this rule?
The phenomena described above is called "the density paradox", and it will be the main reaserch subject of this piece.

In this Paper we will try to estimate the delicate connection between different metrics and 
the amount of errors in a given code. We will try to portray both the typical amount and the typical density
of errors as well defined functions of these metrics, and to understand what measures could be taken to
minimize these risk factors in a software project.
