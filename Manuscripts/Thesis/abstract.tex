The main result of this thesis is Fajita, a practical prototypical tool for the
automatic generation of \NonCitingUse{Java} Fluent APIs from their
specification.

We begin with a theoretical study, that explains why the problem's core lies
with the expressive power of \NonCitingUse{Java} generics.  We show that
automatic generation is possible whenever the specification is an instance of
the set of deterministic context-free languages, a set which contains most
``practical'' languages.

We then advances to present, for the first, an efficient (specifically linear
time) algorithm for generating an automaton, implementable within the framework
of compile time computation of \NonCitingUse{Java}, which recognizes
a given LL(1) language. The generated automaton is time efficient,
spending a constant amount of time on each symbol of the ``input''. Space
requirement is also polynomially bounded. 

Other contributions include a collection of techniques and idioms of the
limited meta-programming possible with \NonCitingUse{Java} generics, and an
empirical measurement demonstrating that the runtime of the ``javac'' compiler
of \NonCitingUse{Java} may be exponential in the program's length, even for
programs composed of a handful of lines and which do not rely on overly complex
use of generics. Another theoretical contribution of this work is the RLLp, an
automaton for recognizing LL(1) grammars that supplies the realtime property.
